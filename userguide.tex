%%
%% User Guide
%%
%% This file should be edited by user
%%

\chapter{User Guide}

You have implemented some system others will use? If you were in
their place what kind of documentation would you like to have in
order to start working?

For complex programs, a tutorial that guides the user through a
typical task showing screenshots of the intermediate states is
desirable.

\section{Types of Bachelor's Theses}

Of course, not all Bachelor's theses require a user guide.


\section{How to use the Thesis Template}

\subsection{Files to Edit}

You should edit the following files (unless you remove some of them,
see Section~\ref{section:removingchapters}):

\begin{itemize}
\item \tt title.tex
\item \tt abstract.tex
\item \tt acronyms.tex
\item \tt introduction.tex
\item \tt concepts.tex
\item \tt relatedwork.tex
\item \tt designapproach.tex
\item \tt implementation.tex
\item \tt results.tex
\item \tt conclusion.tex
\item \tt setupguide.tex
\item \tt userguide.tex
\end{itemize}

\subsection{Removing Chapters} \label{section:removingchapters}

Open the file \texttt{thesis.tex} and remove (or insert a comment)
at the lines with the include-commands, for example:

\begin{table}[ht]
\begin{center}
\begin{tabular}{|l|l|}
  \hline
  Before & After \\
  \hline
  % after \\: \hline or \cline{col1-col2} \cline{col3-col4} ...
  \verb+%%
%% Concepts
%%

\chapter{Concepts} \label{chapter:concepts}

Simulation is the imitation of a real-world process or system over time\cite{wiki:simulation}.
There are numerous variations of this concept, including simulation for training purposes
(such as aircraft simulators), simulation of natural phenomenons for scientific study
or for creating forecast (like weather systems), and simulation simply for entertainment purposes
as can be found most prominently in computer games.

This thesis concerns itself with yet another kind of simulation, which imitates
the operation of a microcontroller and attached peripherals on common desktop computer architectures. More
specifically, \simavr simulates the family of 8-bit \ac{AVR} microcontroller systems
on any modern system supported by \emph{avr-libc}.

In this case, simulation simplifies both the development and debugging of \ac{AVR}
programs by making software engineering staples such as automatic unit tests possible
and vastly easing access to debugging methods and tools such as \ac{GDB} without further
required hardware.

Emulation is another related concept which is only indistinctly different from
simulation. Originally, emulation was first used in 1963 to refer to a type
of simulation which consisted of mixing hardware microcode and software components
to achieve faster simulation speed\cite{building_ibm}. However, this definition is now
largely obsolete. The current concensus seems to be that emulation focuses on imitating
the tangible effects of a system, while simulation concerns itself especially with
a system's inner state.

In this context, \simavr can be considered to be a mixture between a simulator and
emulator; it tries to both maintain approximately realistic execution times while
producing identical execution results as when running on an \ac{AVR} \ac{MCU} as
well as maintaining an accurate internal state at all times.

Emulators usually consist of a \ac{CPU} and memory subsystems which interact with
\ac{IO} device simulations\cite{wiki:emulation}.

The \ac{CPU} subsystem is responsible
for imitating the state and operation of a processor. Specifically, essential components of an \ac{AVR}
processor are simulation of the \ac{SREG} and instruction decoding/processing.
During each simulation cycle, the subsystem decodes an instruction from program memory
and subsequently ``executes'' it by performing semantically equivalent operations on
the host \ac{CPU}. Finally, the \ac{SREG} is updated accordingly.

The memory subsystem not only keeps track of memory contents, but also limits access
to the available memory space, and handles \ac{IO} mapped registers which
trigger interactions with some \ac{IO} device when read or written.

Finally, \ac{IO} devices may be connected to the core subsystems and be programmed
to respond to (and emit) signals.

\simavr already contains the \ac{CPU} and memory subsystems, as well as some
\ac{IO} devices closely associated with \ac{AVR} \acp{MCU} such as an internal \ac{EEPROM},
\acp{ADC}, \ac{IO} ports, timers, and more. \qsimavr will extend upon this existing functionality to
implement several further \ac{IO} modules which interact seamlessly with \simavr, while also offering a
graphical user interface to control the simulation.
+ & \verb+%%
%% Concepts
%%

\chapter{Concepts} \label{chapter:concepts}

Simulation is the imitation of a real-world process or system over time\cite{wiki:simulation}.
There are numerous variations of this concept, including simulation for training purposes
(such as aircraft simulators), simulation of natural phenomenons for scientific study
or for creating forecast (like weather systems), and simulation simply for entertainment purposes
as can be found most prominently in computer games.

This thesis concerns itself with yet another kind of simulation, which imitates
the operation of a microcontroller and attached peripherals on common desktop computer architectures. More
specifically, \simavr simulates the family of 8-bit \ac{AVR} microcontroller systems
on any modern system supported by \emph{avr-libc}.

In this case, simulation simplifies both the development and debugging of \ac{AVR}
programs by making software engineering staples such as automatic unit tests possible
and vastly easing access to debugging methods and tools such as \ac{GDB} without further
required hardware.

Emulation is another related concept which is only indistinctly different from
simulation. Originally, emulation was first used in 1963 to refer to a type
of simulation which consisted of mixing hardware microcode and software components
to achieve faster simulation speed\cite{building_ibm}. However, this definition is now
largely obsolete. The current concensus seems to be that emulation focuses on imitating
the tangible effects of a system, while simulation concerns itself especially with
a system's inner state.

In this context, \simavr can be considered to be a mixture between a simulator and
emulator; it tries to both maintain approximately realistic execution times while
producing identical execution results as when running on an \ac{AVR} \ac{MCU} as
well as maintaining an accurate internal state at all times.

Emulators usually consist of a \ac{CPU} and memory subsystems which interact with
\ac{IO} device simulations\cite{wiki:emulation}.

The \ac{CPU} subsystem is responsible
for imitating the state and operation of a processor. Specifically, essential components of an \ac{AVR}
processor are simulation of the \ac{SREG} and instruction decoding/processing.
During each simulation cycle, the subsystem decodes an instruction from program memory
and subsequently ``executes'' it by performing semantically equivalent operations on
the host \ac{CPU}. Finally, the \ac{SREG} is updated accordingly.

The memory subsystem not only keeps track of memory contents, but also limits access
to the available memory space, and handles \ac{IO} mapped registers which
trigger interactions with some \ac{IO} device when read or written.

Finally, \ac{IO} devices may be connected to the core subsystems and be programmed
to respond to (and emit) signals.

\simavr already contains the \ac{CPU} and memory subsystems, as well as some
\ac{IO} devices closely associated with \ac{AVR} \acp{MCU} such as an internal \ac{EEPROM},
\acp{ADC}, \ac{IO} ports, timers, and more. \qsimavr will extend upon this existing functionality to
implement several further \ac{IO} modules which interact seamlessly with \simavr, while also offering a
graphical user interface to control the simulation.
+ \\
  \verb+\cleardoublepage+ & \verb+\cleardoublepage+ \\
  \verb+%%
%% Related Work
%%

\chapter{Related Work} \label{chapter:relatedwork}

\section{Atmel AVR Studio}
\section{simulavr}
\section{avr-sim}

% TODO
+ & \verb+%%%
%% Related Work
%%

\chapter{Related Work} \label{chapter:relatedwork}

\section{Atmel AVR Studio}
\section{simulavr}
\section{avr-sim}

% TODO
+ \\
  \verb+\cleardoublepage+ & \verb+%\cleardoublepage+ \\
  \verb+%%
%% Design Approach
%%
%% This file should be edited by user
%%

\chapter{Design Approach} \label{chapter:designapproach}

Plugin based, small core, threaded, ...

%%
%% = eof =====================================================================
%%
+ & \verb+%%
%% Design Approach
%%
%% This file should be edited by user
%%

\chapter{Design Approach} \label{chapter:designapproach}

Plugin based, small core, threaded, ...

%%
%% = eof =====================================================================
%%
+ \\
  \verb+\cleardoublepage+ & \verb+\cleardoublepage+ \\
  \hline
\end{tabular}
\end{center}
  \caption{Removing the Chapter \texttt{Related Work}}\label{table:removing}
\end{table}

\subsection{Adding Chapters} \label{section:addingchapters}

Open the file \texttt{thesis.tex} and add an
\verb+\include{+\emph{filename}\verb+}+ command (followed by a
\verb+\cleardoublepage+) at the respective line. Then create a new
file \emph{filename}\verb+.tex+ in the same directory and write the
chapter's contents into it.

\subsection{Adding References} \label{section:addingreferences}

Whenever you \verb+\cite+ something, there must be a respective
entry in any of the included .bib files. To add such an entry, open
the respective bibfile (\eg \texttt{bibfile.bib}) with a text editor
and add the entry according to the bibfile syntax. The reference
will be added after running latex/bibtex/latex on your project.

\subsection{Removing References} \label{section:removingreferences}

Remove all \verb+\cite+ commands to this reference in the text and
the citation will not be listed in the bibliography after running
latex/bibtex/latex on your project. There is no need to remove the
reference from the .bib file.

\subsection{Changing the .bib Files to be used}

In the file \texttt{thesis.tex} there is a line with a
\verb+\bibliography+ command, that lists all used .bib files without
file extensions separated by commas.

\subsection{Troubleshooting}

If you make something wrong, you should get an error at compile
time, except for citation problems, like:

\begin{description}
\item[Adding or removing references does not work:]~~Perhaps there is
a spelling error in some .bib file, this causes the program
\texttt{bibtex} to abort execution leaving the old bibliography
unchanged.

\item[Changing a reference does not work:]~~Some systems only perform
a new bibtex run if there are missing references, delete the
\texttt{.aux} file in order to overcome this problem.

\item[References show a question mark:]~~Bibtex could not find a
matching entry in the bibfile for this reference, check the label
name.

\end{description}



%%
%% = eof =====================================================================
%%
