%%
%% Related Work
%%

\chapter{Related Work} \label{chapter:relatedwork}

Besides \simavr, there are currently several other \ac{AVR} simulator
applications available.

Atmel's \ac{AVR} \ac{IDE}, \emph{Atmel Studio}\footnote{
%
\url{http://www.atmel.com/microsite/atmel_studio6/default.aspx}, accessed 2012-08-31.
%
}, which is based on Microsoft Visual studio, includes tools for \ac{AVR} simulation
and debugging. According to Atmel,
``Simulation supports debug commands such as Run, Break, Reset, Single Step,
Set Breakpoints, and Watch Variables. The I/O, memory and register views are
fully functional using the simulator. [...] With the debuggers connected, Atmel
Studio 6 can present the status of the processor, memories and all communication
and analog interfaces in views that are easy to understand, giving you fast
access to critical system parameters.`` \cite{atmel} However, \emph{Atmel Studio}
is only available on the Windows operating system. Also, it does not supports
simulation of external components such as \ac{LCD} displays.

\emph{simulavr}\footnote{
%
\url{http://www.nongnu.org/simulavr/}, accessed 2012-08-31.
%
} is an open source simulator for \ac{AVR} microcontrollers.

% TODO
