%%
%% Implementation
%%

%%%%%%%%%%%%%%%%%%%%%%%%%%%%%%%%%%%%%%%%%%%%%%%%%%%%%%%%%%%%%%%%%%%%%%%%%%%%%%%%
\chapter{Implementation} \label{chapter:implementation}
%%%%%%%%%%%%%%%%%%%%%%%%%%%%%%%%%%%%%%%%%%%%%%%%%%%%%%%%%%%%%%%%%%%%%%%%%%%%%%%%

We will now examine the actual implementation of \qsimavr and how its individual
parts work together, beginning with the core, the component interface,
a short walkthrough of all components, and finally brief descriptions of the
modifications and additions made to \simavr during the course of this thesis.


%%%%%%%%%%%%%%%%%%%%%%%%%%%%%%%%%%%%%%%%%%%%%%%%%%%%%%%%%%%%%%%%%%%%%%%%%%%%%%%%
\section{\qsimavr Core} \label{section:core}
%%%%%%%%%%%%%%%%%%%%%%%%%%%%%%%%%%%%%%%%%%%%%%%%%%%%%%%%%%%%%%%%%%%%%%%%%%%%%%%%

The core of \qsimavr is located in the \verb|QSimAVR/| subfolder of the source
tree. Its responsibilities include handling the routine tasks of the main window
(including menu and action handling, status bar updates, displaying ``File Open''
dialogs when loading firmware, etc \ldots), loading and subsequently managing
plugins, and actually running the \simavr main loop.

In the following sections, we will discuss the interesting parts of the
\lstinline|SimAVR| and \lstinline|PluginManager| classes. Interested readers
should consult the source code for complete implementation details.


%%%%%%%%%%%%%%%%%%%%%%%%%%%%%%%%%%%%%%%%%%%%%%%%%%%%%%%%%%%%%%%%%%%%%%%%%%%%%%%%
\subsection{\lstinline|SimAVR|} \label{subsection:class_simavr}
%%%%%%%%%%%%%%%%%%%%%%%%%%%%%%%%%%%%%%%%%%%%%%%%%%%%%%%%%%%%%%%%%%%%%%%%%%%%%%%%

The \lstinline|SimAVR| class encapsulates the \lstinline|avr_t| instance. It
runs in its own thread by subclassing \lstinline|QThread| and being started
with the \lstinline|start()| function.

\begin{lstlisting}
class SimAVR : public QThread
{
    Q_OBJECT

public:
    SimAVR();
    virtual ~SimAVR();

    void load(const QString &filename);
    void run();

public slots:
    void pauseSimulation();
    void stopSimulation();
    void attachGdb();

private:
    avr_t *avr;

    /* [...] */
};
\end{lstlisting}

\lstinline|SimAVR| can load firmware, initialize the \lstinline|avr_t| instance,
and execute its main loop:

\begin{lstlisting}
int state;
uint8_t i = 0;
do {
    state = avr_run(avr);
    if (i++ == 0) {
        QCoreApplication::processEvents();
    }
} while (state != cpu_Done && state != cpu_Crashed);
emit simulationStateChanged(Done);
\end{lstlisting}

The loop itself is very similar to ones we have already seen in section
\ref{section:simavr_example_walkthrough}; additionally, since we heavily use
queued signals, we need to call \lstinline|processEvents| to allow the internal
Qt event handler to forward signals. However, profiling sessions have shown
\lstinline|processEvents| to require an unacceptable amount of processing time
when called each loop iteration - it appears that every call will require at least
one mutex lock and unlock cycle. For performance reasons, we therefore limited
event processing to occur only once every $2^8$ iterations.\footnote{
%
This is also somewhat less than ideal when using \ac{GDB} for short steps, since
results will only be visible with an average delay of $2^7$ iterations.
%
}
Changes in simulation state are in turn
emitted as signals (which are then processed by \lstinline|MainWindow| and
displayed in the status bar).

\lstinline|SimAVR| also provides functions for pausing, unpausing and stopping the
simulation, and for preparing \simavr to listen to incoming \ac{GDB} connections.
Each of these operations also results in changes to the simulation state.

Whenever a firmware is loaded or unloaded, signals are emitted and processed
by \lstinline|PluginManager| to respectively connect and disconnect components
from the \lstinline|avr_t| instance.


%%%%%%%%%%%%%%%%%%%%%%%%%%%%%%%%%%%%%%%%%%%%%%%%%%%%%%%%%%%%%%%%%%%%%%%%%%%%%%%%
\subsection{\lstinline|PluginManager|}
%%%%%%%%%%%%%%%%%%%%%%%%%%%%%%%%%%%%%%%%%%%%%%%%%%%%%%%%%%%%%%%%%%%%%%%%%%%%%%%%

The plugin manager

\begin{itemize}
\item Can load components from shared libraries,
\item Creates \ac{MDI} sub-windows within the main \lstinline|QMdiArea| as needed,
\item Manages their connections to \lstinline|avr_t| instances as plugins are
      enabled or disabled and simulation is restarted or loaded with a new firmware
      file,
\item Turns \ac{VCD} tracing on depending on the selections made in component
      settings, and
\item Saves and restores component settings.
\end{itemize}

Loading plugins is implemented such that files in a specific directory (which can be
set during installation, see section \ref{section:setup_qsimavr}) are iterated
one by one. If loading a file fails, an error message is printed to the log targets
and loading continues with the next file. Each component must adhere to the
component interface (see section \ref{subsection:component_interface}) and contain
the plugin registration function.
It is also possible to keep components
in different directories, only placing symlinks (symbolic links) into
\lstinline|PLUGINDIR|.

\begin{lstlisting}
QLibrary lib(filename);
RegisterFunction registerPlugin = (RegisterFunction)lib.resolve(PUBLISH_FNAME);

QSharedPointer<ComponentFactory> factory(registerPlugin());
\end{lstlisting}

Component registration uses the \lstinline|QLibrary| class to load the shared library,
and then creates a factory instance by calling the \lstinline|registerPlugin|
function common to each component.

The factory then creates the actual logic and \ac{GUI} parts. If a \ac{GUI} part
exists, a \lstinline|QMdiSubWindow| is created containing the component's widget.
The logic part is moved to the \lstinline|SimAVR| thread.

Whenever \lstinline|firmwareLoaded| or \lstinline|firmwareUnloaded| signal is sent
by \lstinline|SimAVR|, the logic part of components is, respectively, connected
or disconnected from the \lstinline|avr_t| instance.

Component settings are saved and loaded using the Qt's \lstinline|QSettings|
class, which provides a simple and cross-platform method of handling settings
files.


%%%%%%%%%%%%%%%%%%%%%%%%%%%%%%%%%%%%%%%%%%%%%%%%%%%%%%%%%%%%%%%%%%%%%%%%%%%%%%%%
\subsection{Component Interface} \label{subsection:component_interface}
%%%%%%%%%%%%%%%%%%%%%%%%%%%%%%%%%%%%%%%%%%%%%%%%%%%%%%%%%%%%%%%%%%%%%%%%%%%%%%%%

Each component must implement the component interface defined in
\verb|QSimAVR/component.h|. It consists of factory, \ac{GUI} and logic classes,
and a \lstinline|registerPlugin| function which returns a factory instance. Let's
take a look at each of these:

\begin{lstlisting}
template<typename Interface>
class Factory {
public:
    Factory() { }
    virtual Interface create() = 0;
    virtual ~Factory() { }

private:
    Q_DISABLE_COPY(Factory)
};

typedef Factory<Component> ComponentFactory;
\end{lstlisting}

The factory is a generic class which simply provides a pure abstract \lstinline|create| function
to create, initialize and return an object. In \qsimavr, we only use the instatiated
version \lstinline|ComponentFactory|, which returns a \lstinline|Component|.

\lstinline|Q_DISABLE_COPY| prevents copying a class by declaring the copy and
assignment constructors in the private section:

\begin{lstlisting}
#define Q_DISABLE_COPY(Class) \
    Class(const Class &); \
    Class &operator=(const Class &);
\end{lstlisting}

In practice, \lstinline|ComponentFactory::create| constructs all parts of a component, connects
their signals and slots\footnote{
%
It is particularly important to realize that these must use queued signal
delivery, since the logic and \ac{GUI} parts do not reside in the same thread.
Specify \lstinline|Qt::QueuedConnection| in \lstinline|QObject::connect()|.
%
} and packs them into a \lstinline|Component|, which is simply a struct containing
smart pointers to the component parts:

\begin{lstlisting}
struct Component
{
    QSharedPointer<ComponentGui> gui;
    QSharedPointer<ComponentLogic> logic;
};
\end{lstlisting}

The \ac{GUI} part of a component provides a method for the user to interact with
the plugin by both displaying component state and allowing user input.

\begin{lstlisting}
class ComponentGui
{
public:
    ComponentGui() { }

    virtual QWidget *widget() = 0;
    virtual ~ComponentGui() { }

private:
    Q_DISABLE_COPY(ComponentGui)
};
\end{lstlisting}

Its interface is very simple; communication with the logic part is handled internally
by signals, and user interaction is handled by the widget itself. The
\lstinline|QWidget| is displayed as the central widget in a \lstinline|QMdiSubWindow|.
Most of the time, the functionality provided by a \ac{GUI} component is simple enough
such that it can inherit both from \lstinline|ComponentGui| and \lstinline|QWidget|
itself\footnote{
%
Therefore, \lstinline|ComponentGui::widget| can simply return the
\lstinline|this| pointer.
%
}.

The logic part encapsulates a component's internal state; it also reacts to signals,
communicates with it's \ac{GUI}, and is responsible for accurately simulating
the behavior of the actual component.

\begin{lstlisting}
class ComponentLogic : public QObject
{
public:
    ComponentLogic(QObject *parent = NULL);
    virtual ~ComponentLogic() { }

    virtual void wire(avr_t *avr);
    virtual void unwire();
    virtual void enableVcd(bool vcdEnabled);

protected:
    virtual void wireHook(avr_t *avr) = 0;
    virtual void unwireHook() = 0;
    virtual void resetHook() { }

protected:
    bool connected;
    bool vcdEnabled;

    avr_vcd_t vcdFile;

    struct component_io_t {
        avr_io_t io;
        ComponentLogic *instance;
    } io;

private:
    static void resetHookPrivate(avr_io_t *io);

private:
    Q_DISABLE_COPY(ComponentLogic)
};
\end{lstlisting}

Common functionality is handled in \lstinline|ComponentLogic| itself by implementing
it in the public function, which then calls a private hook at an appropriate time
(for example, see \lstinline|wire| and \lstinline|wireHook|).

\lstinline|wire| connects a component to the \lstinline|avr_t| instance. Most of
the time, this involves creating internal \acp{IRQ}, connecting them to the
desired target \acp{IRQ}, setting up notification callbacks, and defining the
signals traced by a \ac{VCD}.

Callback functions require a little extra work because \simavr is written in plain
C and has no notion of classes or member functions. A simple solution to this problem
is to register a static class function as the callback, pass \lstinline|this| as
the callback \lstinline|param|, and let the static function execute the actual member
function. A typical example (taken from the temperature component):

\begin{lstlisting}
void TemperatureLogic::wireHook(avr_t *avr)
{
    /* [...] */
    avr_irq_register_notify(irq + IRQ_TEMP_DQ, TemperatureLogic::pinChangedHook, this);
    /* [...] */
}

void TemperatureLogic::pinChangedHook(avr_irq_t *, uint32_t value, void *param)
{
    TemperatureLogic *p = (TemperatureLogic *)param;
    p->pinChanged(value);
}
\end{lstlisting}

\lstinline|enableVcd| simply toggles \ac{VCD} traces on and off. A default implementation
is provided in \lstinline|ComponentLogic| itself.

Again, supporting \simavr
\ac{VCD} traces has an interesting quirk; as explained in section \ref{section:vcd_files},
\ac{VCD} files depend upon \simavr cycle timers\footnote{
%
See section \ref{section:cycle_timers}.
%
}, which are transient and deleted entirely when \lstinline|avr_reset| is called.
We therefore require a way to recreate cycle timers on each reset. This functionality
is provided the \lstinline|avr_io_t| module (section \ref{subsection:avr_io_t}),
or more specifically, its \lstinline|reset| callback. Each component is registered
with the \lstinline|avr_t| core as such a module. On each reset event,
\ac{VCD} connections are automatically restored, and the (optional)
\lstinline|ComponentLogic::resetHook| of every component is called for any
custom setup steps.

The final piece of the \lstinline|Component| interface is the registration function.
Plugins must declare themselves using the \lstinline|PUBLISH_PLUGIN| macro,
which defines a C function returning a factory instance:

\begin{lstlisting}
#define PUBLISH_PLUGIN(factory) \
    extern "C" { \
        ComponentFactory *registerPlugin() { return new factory; } \
    }
\end{lstlisting}


%%%%%%%%%%%%%%%%%%%%%%%%%%%%%%%%%%%%%%%%%%%%%%%%%%%%%%%%%%%%%%%%%%%%%%%%%%%%%%%%
\section{\acf{TWI} Component} \label{section:component_twi}
%%%%%%%%%%%%%%%%%%%%%%%%%%%%%%%%%%%%%%%%%%%%%%%%%%%%%%%%%%%%%%%%%%%%%%%%%%%%%%%%

We now turn our attention to the actual component implementations. The \ac{TWI}
component is actually not a component as defined by the component interface in
section \ref{subsection:component_interface}; rather, it is a library providing
helper functions for other components acting as a slave
using the \ac{TWI} protocol\footnote{
%
The \ac{TWI} component compiles into a static library which can be included
by any other component requiring it.
%
}.

To prevent duplicate work and the \ac{NIH} syndrome\footnote{
%
For programmers, laziness is a virtue.
%
}, the \ac{TWI} implementation is an adaption of the work done by the \simavr
author in the \verb|i2c_eeprom| component. The \ac{TWI} slave logic was extracted
into a separate library because it is needed by both the \ac{EEPROM}
(section \ref{section:component_eeprom}) and \ac{RTC} (section \ref{section:component_rtc})
components.

We simply connect ourselves to the \lstinline|TWI_IRQ_MISO| and \lstinline|TWI_IRQ_MOSI|
\acp{IRQ}, process incoming and outgoing messages according to the \ac{TWI}
protocol, and notify interested parties when we've either received a complete message,
or we need to send data to the \ac{TWI} master.

For that purpose, a \lstinline|TwiSlave| interface is defined which components
using \lstinline|TwiComponent| must implement.

\begin{lstlisting}
class TwiSlave
{
public:
    virtual uint8_t transmitByte() = 0;
    virtual bool matchesAddress(uint8_t address) = 0;
    virtual void received(const QByteArray &data) = 0;
};
\end{lstlisting}

\lstinline|transmitByte| is called by the \ac{TWI} component whenever the master
begins a read transaction and returns the next byte to send.
\lstinline|matchesAddress| returns true, \ac{iff} the slave address matches
the address sent by the master or if the general call address (\lstinline|0x0|)
has been sent.
\lstinline|received| is called with the received data once a write transaction
completes, and allows the slave to react to a transmission.

%%%%%%%%%%%%%%%%%%%%%%%%%%%%%%%%%%%%%%%%%%%%%%%%%%%%%%%%%%%%%%%%%%%%%%%%%%%%%%%%
\section{\acf{EEPROM} Component} \label{section:component_eeprom}
%%%%%%%%%%%%%%%%%%%%%%%%%%%%%%%%%%%%%%%%%%%%%%%%%%%%%%%%%%%%%%%%%%%%%%%%%%%%%%%%

The \ac{EEPROM} uses the \ac{TWI} component briefly described in section \ref{section:component_twi}.
It has been implemented using the datasheet \cite{microchip01} as a reference.
The simulated \ac{EEPROM} consists of 1 kbit of storage and is controlled over a
\ac{TWI} bus.

The implementation itself is very simple since the \ac{TWI} logic is encapsulated
by the \ac{TWI} component. We can therefore look at the \ac{EEPROM} from a very
high level, and are only interested in keeping track of the \ac{AP} and the state
of the internal memory. Reading from the \ac{EEPROM} returns the byte pointed to
by the \ac{AP} and then increments the \ac{AP}. Similarly, writing data section
writes it to the \ac{AP} and subsequently increments it.

\begin{table}[ht]
\centering
\begin{tabular}{l|l}

Pin/\ac{IRQ}        & Function\\

\hline

\ac{TWI} \acp{IRQ}  & \ac{TWI} connections

\end{tabular}
\caption{\ac{EEPROM} Wiring}
\label{tab:wiring_eeprom}
\end{table}

This functionality is achieved by implementing the \lstinline|TwiSlave| methods
as follows (edited for clarity):

\begin{lstlisting}
uint8_t EepromLogic::transmitByte()
{
    return eeprom[incrementAddress()];
}

void EepromLogic::received(const QByteArray &data)
{
    addressPointer = data[0];
    for (int i = start; i < data.size(); i++) {
        eeprom[incrementAddress()] = data[i];
    }
}

bool EepromLogic::matchesAddress(uint8_t address)
{
    return ((address & EEPROM_MASK) == (EEPROM_ADDR & EEPROM_MASK) ||
            (address & 0xfe) == 0x00);
}
\end{lstlisting}

The \lstinline|QByteArray| is used to represent the byte storage internally as it
provides both the convenience of many helper functions, and the advantages
of raw memory arrays through \lstinline|QByteArray::data|.


%%%%%%%%%%%%%%%%%%%%%%%%%%%%%%%%%%%%%%%%%%%%%%%%%%%%%%%%%%%%%%%%%%%%%%%%%%%%%%%%
\section{\acf{RTC} Component} \label{section:component_rtc}
%%%%%%%%%%%%%%%%%%%%%%%%%%%%%%%%%%%%%%%%%%%%%%%%%%%%%%%%%%%%%%%%%%%%%%%%%%%%%%%%

The \ac{RTC} component consists of a 56 byte memory area\footnote{
%
8 of which are used to represent clock state, with the rest usable as generic
storage.
%
}, communications over a \ac{TWI} bus, and timekeeping logic which updates its
clock registers once per second \cite{maxim01}.

\begin{table}[ht]
\centering
\begin{tabular}{l|l}

Pin/\ac{IRQ}        & Function\\

\hline

\ac{TWI} \acp{IRQ}  & \ac{TWI} connections

\end{tabular}
\caption{\ac{RTC} Wiring}
\label{tab:wiring_rtc}
\end{table}

Internally, the implementation is very similar to the \ac{EEPROM} examined in section
\ref{section:component_eeprom}; so much in fact, that only differences to the \ac{EEPROM}
will be discussed here. Writing and reading from the memory storage is identical.

The first 7 bytes represent the clock's date and time in \ac{BCD} format and need
to be updated periodically to reflect the passing of time. Once again,
\simavr's cycle timers (discussed in section \ref{section:cycle_timers}) came in
handy to achieve the desired functionality. A callback is registered to occur
once every second (of simulated time), which updates the the date
and time bytes. Handling of leap years is simplified by relying on Qt's
\lstinline|QDate| class.

The square wave pin of the actual component is \emph{not} implemented.

%%%%%%%%%%%%%%%%%%%%%%%%%%%%%%%%%%%%%%%%%%%%%%%%%%%%%%%%%%%%%%%%%%%%%%%%%%%%%%%%
\section{\acf{GLCD} Component} \label{section:component_glcd}
%%%%%%%%%%%%%%%%%%%%%%%%%%%%%%%%%%%%%%%%%%%%%%%%%%%%%%%%%%%%%%%%%%%%%%%%%%%%%%%%

The \ac{GLCD} component simulates a 128x64 Graphic \ac{LCD} display, based internally
on two NT7108 controller chips. Each of these has 512 bytes of display \ac{RAM},
and three address pointers (storing the horizontal pixel address, the vertical page
address, and the first page displayed at the top edge of the screen)
\cite{winstar01, samsung01, neotec01}.

The touchscreen is also included within this component. The initial idea was to
implement it in a separate component which would be displayed in a transparent
\ac{MDI} window the same size as the \ac{GLCD} screen and could therefore be
layered on top of it. However, limitations in \ac{MDI} transparency handling
(which either drew the entire window transparent including borders and title bar
or everything opaque) put an end to that idea.

\begin{table}[ht]
\centering
\begin{tabular}{l|l}

Pin/\ac{IRQ}        & Function \\

\hline

PORTA0-7            & Data Pins\\
PORTE2              & \acf{CS1}\\
PORTE3              & \acf{CS2}\\
PORTE4              & \acf{RS}\\
PORTE5              & \acf{RW}\\
PORTE6              & \acf{E}\\
PORTE7              & \acf{RST}\\
\ac{ADC}0 \ac{IRQ}  & Touchscreen X Axis\\
\ac{ADC}1 \ac{IRQ}  & Touchscreen Y Axis

\end{tabular}
\caption{\ac{GLCD} Wiring}
\label{tab:wiring_glcd}
\end{table}


This is one of the most complex components, as it both requires nontrivial steps
to decode instructions sent by the \ac{AVR} core, and contains a \ac{GUI} part
which must handle both rudimentary pixel-wise drawing and mouse input.
The \ac{GLCD} reacts to edges of the \ac{E} pin. On the rising edge, the pin state
is saved into the command buffer. If the command triggers a read operation\footnote{
%
By ``read operation'', we mean that the \ac{AVR} core wants to read from
the \ac{GLCD} component.
%
}, we queue a cycle timer which will write the requested data to the data pins
upon completion of 320 \ac{ns} data delay time. Note that display read data first passes an
intermediate read buffer before it is actually output to the data pins. This
simulates latching data into the output register which actually occurs in the real
hardware\footnote{
%
Status reads are not affected by this.
%
}.

Write operations are handled on the falling \ac{E} edge. After decoding the
buffered command, current state of the data pins is used to perform the operation.

Each controller chip is simulated by an instance of the \lstinline|NT7108| class.
These are coordinated by the \lstinline|GlcdLogic| class, which forwards signals
according to the current state of the \ac{CS1} and \ac{CS2} pins. Whenever an
operation causes the visible state of the display to change, a signal is emitted
which is in turn handled by the \ac{GUI} part of the component.

The frontend to the \ac{GLCD} is implemented using a \lstinline|QGraphicsScene|,
which simplifies routine tasks such as drawing and receiving user input. Each pixel
is represented by a \lstinline|QGraphicsRectItem|; if it is turned off, we use
a \lstinline|Qt::green| brush, otherwise it is displayed as \lstinline|Qt::black|.
Now, when a page of the \ac{GLCD} display \ac{RAM} is changed, all eight affected
pixels are updated.

The scene also receives mouse events and emits these as signals. These are received
by the logic part and forwarded to an instance of the \lstinline|Touchscreen| class,
where the affected coordinates are buffered. These can be requested by the \ac{AVR}
core by querying \ac{ADC}0 and \ac{ADC}1, which are respectively responsible
for the X and Y axis. An \ac{ADC} value of 2700 millivolts corresponds approximately
to the upper and right edges of the screen. The voltages scale linearly until
reaching 0 volts at the lower and left edges. If no ``touch'' is currently active,
the coordinates are reset to the neutral lower-left corner.


%%%%%%%%%%%%%%%%%%%%%%%%%%%%%%%%%%%%%%%%%%%%%%%%%%%%%%%%%%%%%%%%%%%%%%%%%%%%%%%%
\section{\acf{LCD} Component} \label{section:component_lcd}
%%%%%%%%%%%%%%%%%%%%%%%%%%%%%%%%%%%%%%%%%%%%%%%%%%%%%%%%%%%%%%%%%%%%%%%%%%%%%%%%

This component simulates a 2x16 \ac{LCD} display based on the HD44780
chip \cite{hitachi01, samsung02, winstar02}. A substantial part of the implementation
was taken from the \verb|hd44780| example part included with \simavr.

\begin{table}[ht]
\centering
\begin{tabular}{l|l}

Pin/\ac{IRQ}        & Function \\

\hline

PORTC4-7            & Data Pins\\
PORTC2              & \acf{RS}\\
PORTC3              & \acf{E}

\end{tabular}
\caption{\ac{LCD} Wiring}
\label{tab:wiring_lcd}
\end{table}

Communication is very similar to the \ac{GLCD} exmined in section
\ref{section:component_glcd} and will not be discussed further in this section.
Font selection, custom fonts and cursor display are not implemented, allowing us to use
a simple \lstinline|QTextBrowser| to display \ac{LCD} output.

%%%%%%%%%%%%%%%%%%%%%%%%%%%%%%%%%%%%%%%%%%%%%%%%%%%%%%%%%%%%%%%%%%%%%%%%%%%%%%%%
\section{\acfp{LED} \& Buttons Component}
%%%%%%%%%%%%%%%%%%%%%%%%%%%%%%%%%%%%%%%%%%%%%%%%%%%%%%%%%%%%%%%%%%%%%%%%%%%%%%%%

The 86 buttons and \acp{LED} of the BIGAVR6 board are bundled into this simple
component.

\begin{table}[ht]
\centering
\begin{tabular}{l|l}

Pin/\ac{IRQ}        & Function \\

\hline

PORTA0-7            & \acp{LED} \& Buttons\\
PORTB0-7            & \acp{LED} \& Buttons\\
PORTC0-7            & \acp{LED} \& Buttons\\
PORTD0-7            & \acp{LED} \& Buttons\\
PORTE0-7            & \acp{LED} \& Buttons\\
PORTF0-7            & \acp{LED} \& Buttons\\
PORTG0-5            & \acp{LED} \& Buttons\\
PORTH0-7            & \acp{LED} \& Buttons\\
PORTJ0-7            & \acp{LED} \& Buttons\\
PORTK0-7            & \acp{LED} \& Buttons\\
PORTL0-7            & \acp{LED} \& Buttons

\end{tabular}
\caption{\acp{LED} \& Buttons Wiring}
\label{tab:wiring_ledbuttons}
\end{table}

Each \lstinline|QPushButton| also doubles as a \ac{LED} by changing its background
color to represent \ac{LED} state. The implementation assumes that pull-ups
are enabled on all ports, and pulls pins to high when a button is released.
The implementation is otherwise farily trivial and there is not much of interest
to be mentioned.

%%%%%%%%%%%%%%%%%%%%%%%%%%%%%%%%%%%%%%%%%%%%%%%%%%%%%%%%%%%%%%%%%%%%%%%%%%%%%%%%
\section{Temperature Sensor Component}
%%%%%%%%%%%%%%%%%%%%%%%%%%%%%%%%%%%%%%%%%%%%%%%%%%%%%%%%%%%%%%%%%%%%%%%%%%%%%%%%

The DS1820 temperature sensor contains a 9 byte scratchpad memory, a 2 byte \ac{EEPROM},
and uses a 1-wire protocol to communicate with the \ac{AVR} core \cite{maxim02}.

\begin{table}[ht]
\centering
\begin{tabular}{l|l}

Pin/\ac{IRQ}        & Function \\

\hline

PORTG0              & 1-Wire Data\\
DDRG                & 1-Wire Data Direction

\end{tabular}
\caption{Temperature Sensor Wiring}
\label{tab:wiring_temperature}
\end{table}

This component surprised us with the unexpected complexity of the 1-wire protocol.
Implementation was further complicated by the difficulties we've encountered
caused by the implementation of \ac{IO} pin levels
in \simavr, which does not know how to to behave correctly with a connected
component; for example, \simavr blindly sets pin levels, even if the \ac{IO} port
is set to input mode. Unfortunately, we had to use fairly unpleasant methods to
counteract that.

Let's recall how \simavr handles a write to an \ac{IO} port. \simavr has no
notion of any connected components; there is no way to tell it whether the component
has a pull-up resistor nor whether the component is currently transmitting high
or low. When \simavr receives a \ac{IO} port write, it also does not know
if it is coming from the \ac{AVR} core, or from an external component.

\begin{lstlisting}
static void avr_ioport_write(struct avr_t * avr, avr_io_addr_t addr, uint8_t v, void * param)
{
    avr_ioport_t * p = (avr_ioport_t *)param;
    avr_core_watch_write(avr, addr, v);

    for (int i = 0; i < 8; i++)
        avr_raise_irq(p->io.irq + i, (v >> i) & 1);
    avr_raise_irq(p->io.irq + IOPORT_IRQ_PIN_ALL, v);
}
\end{lstlisting}

As you can see, the \lstinline|IOPORT| \acp{IRQ} are triggered on every port write.
This of course presents us with a problem; even if the port is set to input\footnote{
%
A port is set to input by clearing the \ac{DDR} pin.
%
}, every port write will be propagated to the \ac{IRQ} representing pin state. The component
not only has to ignore this invalid level change, it also has to revert it.
Since setting a new \ac{IRQ} level is not possible within the \ac{IRQ} callback function,
we've fallen back on scheduling a cycle timer with a duration of zero to correct
the \ac{IRQ} value. Of course, all of this only serves to complicate component
code and to hide its actual purpose.

Even after ignoring the issue of erroneous pin values (which is solved by ignoring
all pin changes not set by the component itself), the temperature component
still has to simulate the behavior of its internal pull-up. We have achieved this
by listening for changes of the \ac{DDR} pin. When the port is switched to input mode,
we pull the 1-wire pin high manually. Upon switching to output mode, \simavr
automatically restores the pin level appropriately.

Having discussed the encountered difficulties and the employed solutions, we
can now turn our attention to the actual protocol implementation located in the
\lstinline|DS1820| class. 1-wire is a time-based protocol, again pointing to the
use of cycle timers.

By saving the cycle of the most recent pin edge and comparing
it with the current cycle, we can easily calculate the low- and high periods.
Receiving a low pulse of a minimum of 480 \ac{us} resets the state machine,
regardless of the state we are currently in. After a reset, the DS1820
expects a ROM command, which is often followed by a function command. Either of
these can involve both reads and writes. Details of the protocol are excellently
demonstrated in the datasheet \cite{maxim02}.

The DS1820 has two memory areas - a transient scratchpad, and a 2 byte \ac{EEPROM}
for permanent storage of alarm levels. Both can be accessed by the \ac{AVR} core
using specific command sequences. On each scratchpad write access, the \ac{CRC}
sum stored in the \nth{9} byte is updated using the \lstinline|crc8| function\footnote{
%
Based on code by Colin O'Flynn.
%
}. Changes to either area are emitted as signals to the \ac{GUI} part, which
displays the memory in an editable hex editor widget\footnote{
%
The used widget is QHexEdit2 by Winfried Simon. The source code is available
at \url{http://code.google.com/p/qhexedit2/}.
%
}. If the user changes memory areas, these are in turn signalled to the
\lstinline|DS1820| class, which updates the corresponding memory areas.

Triggering a temperature conversion in this implementation does nothing, since the
current temperature can always be written directly into the scratchpad memory by the user.

%%%%%%%%%%%%%%%%%%%%%%%%%%%%%%%%%%%%%%%%%%%%%%%%%%%%%%%%%%%%%%%%%%%%%%%%%%%%%%%%
\section{Changes and additions to \simavr}
%%%%%%%%%%%%%%%%%%%%%%%%%%%%%%%%%%%%%%%%%%%%%%%%%%%%%%%%%%%%%%%%%%%%%%%%%%%%%%%%

While working on \qsimavr, we have also spent a very substantial amount of time
with \simavr itself; we have not only worked on adding new features, but also encountered
several misbehaviors which we have often been able to correct after extensive debugging.
In this section, we will discuss the more interesting changing to \simavr which
we have contributed during the course of this thesis\footnote{
%
A complete list of contributions is available in the simavr git commit log. For
instructions on how to retrieve the source code see section \ref{section:setup_simavr}.
%
}.

The \verb|atmega1280| core definition has been amended to include the upper \ac{ADC}
differential channels, timers 4 and 5, and \ac{UART} 2 and 3, and raised
\lstinline|MAX_IOs| to include all required \ac{IO} registers.

We've added support for \ac{GDB} data watchpoints (which has already been
partly described in section \ref{section:gdb_support}).

An error preventing correctly reading and writing \ac{SREG} from \ac{GDB} was fixed
by de- and reconstructing \ac{SREG} on each access.

Inaccurate sleep timing caused by short cycle timer periods were solved by
accumulating requested sleep times until a certain threshold has been reached
(see also section \ref{section:mainloop}).

We prevented spurious \ac{CPU} wakeups while stopped by explicitly checking for
\lstinline|avr->state == cpu_Sleeping|
(instead of \lstinline|avr->state != cpu_Running|).

Although many parts of \simavr could already be included into C++ projects
without issues, some were still missing the required preprocessor instructions.
We've amended these such that every header can now be used by C++ applications.

Inconsistent handling of \lstinline|avr_terminate| has been improved by never
calling it within \simavr. \lstinline|avr_t| destruction must \emph{always}
be done by the user. This interfered with cleanup routines which expected
\lstinline|avr_t| to stay valid while in fact they could be destructed when
the \ac{AVR} core sleeps while interrupts have been turned off.

Support has been added for interrupts which (contrary to the norm) do not clear
their ``raised'' flag bit when entering the \ac{ISR}. In particular, this
concerns the \ac{TWI} interrupt flag (TWINT) and caused problems during the
\ac{TWI} simulation.

We've corrected an issue which would incorrectly identify an ``sleeping with
interrupts off'' state and terminate simulation. This occurred when an interrupt
was raised and a SLEEP instruction processed before the interrupt could be
serviced due to \lstinline|pending_wait|.

Logging output, which was previously printed to \lstinline|stderr| and
\lstinline|stdout| without distinction, is now categorized into log levels with
increasing verbosity \lstinline|LOG_ERROR|, \lstinline|LOG_WARNING|, and
\lstinline|LOG_TRACE|. The log level can be selected by passing \verb|-v|
when running the \simavr binary, or setting \lstinline|avr->log|.

We've fixed a bug in timer handling related to TCNT \ac{IO} register writes that could
result in an infinite cycle timer loop.

A situation in which the \ac{IO} port \ac{IRQ} would miss level changes was fixed by
removing manual filtering and relying instead on the \lstinline|IRQ_FLAG_FILTERED|
behavior.

Finally, \lstinline|IO| port \acp{IRQ} now restore themselves to PORT
the register value when \ac{DDR} is switched to output mode.
