%%
%% Introduction
%%

\chapter{Introduction} \label{chapter:introduction}

Developing complex applications is an extremely difficult task, even more so when
the targeted platform is as close to the hardware as microcontrollers. Unlike
typical computer programs, programming for a microcontroller is done at a very level,
requiring knowledge of the exact hardware specifications and using strict protocols
to correctly communicate with external periphery. These protocols often require
sending signals a specific timing bounds (these are often in the micro- and nanosecond
range), and upon failure to do so proceed with undefined behavior instead of
printing a (hopefully) informative error message or throwing a well-defined
exception.

Debugging on the other hand is complicated by the fact that programs are executed
not on the development host, but on a separate piece of hardware.
Debugging options include using \lstinline|printf| to output text
over a \ac{UART} interface, or toggling \acp{LED} at specific points of interest
within the program.
While debuggers may be used by connecting a \ac{JTAG} device such as the Atmel
JTAGICE mkII,
these are often expensive\footnote{
%
And specifically, not available to students in the \ac{TU} Vienna Microcontroller course.
%
}.

The answer to these problems lies in simulation. Binary \ac{ELF} files can be
interpreted by a simulator to mimic the state and actions of an \ac{AVR} controller
loaded with the same \ac{ELF} firmware file. This program can not only run on the development
host, but also implement sophisticated development aids such as debugging support
using standard tools, producing execution and signal traces, and replaying recorded
signal sequences. Simulation simplifies development of \ac{AVR} programmers and improves
the quality of produced artifacts while simultaneously lower the entry barrier
for new programmers by allowing them to debug without acquiring expensive
hardware (or even doing initial development without posessing a hardware microcontroller
at all). Unit testing and regression checks can be created that run automatically
on the development machine, and environments may be simulated by recording and then
replaying signal sequences at will.

The aim of this thesis is to produce such a tool for \ac{AVR} microntrollers,
specifically the \verb|atmega1280| processor running on a frequency of 16 \ac{MHz}
on a MikroElektronika BIGAVR6 Development System\footnote{
%
\url{http://www.mikroe.com/bigavr/}, accessed 2012-09-01.
%
}, based on the \ac{AVR} core simulation provided by \simavr\footnote{
%
\url{https://github.com/buserror-uk/simavr}, accessed 2012-09-01.
%
}. External periphery of the BIGAVR6 board will also be simulated, including
the \ac{LCD} and \ac{GLCD} displays, the \ac{TWI} \ac{EEPROM} and \ac{RTC},
on-board \acp{LED} and push buttons connected to the \ac{IO} ports, a touchscreen
attached to the \ac{GLCD}, and a temperature sensor using the 1-wire protocol.

This document is structured as follows: Chapter \ref{chapter:introduction} briefly
introduces the topic of this work and its objectives.

% TODO: Concepts?

Chapter \ref{chapter:relatedwork} summarizes the current state of related \ac{AVR}
simulation software.

The internals of \simavr are examined in-depth in chapter \ref{chapter:simavr}.

Chapter \ref{chapter:designapproach} discusses the project requirements, and
how the chosen design approach will achieve these as well as important decisions
made during the development process.

Chapter \ref{chapter:implementation} walks through the implementation of the
\qsimavr core as well as the provided components.

Results and potential for further work are presented in chapter \ref{chapter:results}.

Appendix \ref{chapter:setup} provides a setup guide for getting up and running
with both \simavr and \qsimavr.

Finally, appendix \ref{chapter:user_guide} contains a \qsimavr user guide
and brief tutorial for debugging \ac{AVR} programs with \qsimavr and \ac{GDB}.
