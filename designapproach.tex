%%
%% Design Approach
%%

\chapter{Design Approach} \label{chapter:designapproach}

Having covered \simavr in-depth, we now turn our attention to the program developed
during the course of this thesis, namely \qsimavr.

\section{Project Outline} \label{section:project_outline}

The aim was to provide an application that could facilitate easy, reliable and
useful simulation not only of the \ac{AVR} core, but also of various external
components (such as \ac{LCD} displays) and their interactions. Specifically,
the following set of components should be simulated for the \verb|atmega1280|
\ac{MCU}, roughly based on the BIGAVR6 board by MikroElektronika\footnote{
%
http://www.mikroe.com/products/view/322/bigavr6-development-system/
%
}:

\begin{itemize}
\item A standard 2x16 character \ac{LCD} display based on the HD44780 chip
\item An external 1 Kbit \ac{EEPROM} connected to the \ac{MCU} via \ac{TWI}
\item A DS1820 Temperature Sensor using 1-wire communication
\item The DS1307 \ac{RTC}, also connected via \ac{TWI}
\item All \acp{LED} connected to \verb|PORT| pins
\item Likewise, all push buttons connected to \verb|PORT| pins
\item A 128x64 Graphic \ac{LCD} display containing two chips based on the NT7108 controller
\item A touch-sensitive panel (usually attached to the \ac{GLCD}
\end{itemize}

The MMC/SDcard reader which is present on the BIGAVR6 board as well as
other external components\footnote{
%
Among others, the SmartMP3 board, a Serial Ethernet board, a \ac{UART} based
Bluetooth board, and an external MMC/SDcard reader.
%
} used by the Microcontroller course at the \ac{TU} Vienna are currently \emph{not} included (these are all good
candidates for further work though).

These components should be simple to connect and disconnect from a running \ac{AVR}
simulation. They should also offer development assistance in the form of \ac{VCD}
optional traces of relevant signals. Graphical output of component state should
be presented to the user, and user input should be forwarded to the simulation.

The frontend should offer at least rudimentary control of firmware
loading, \ac{GDB} connection and simulation execution.

Additionally, bonus points are rewarded for cross-platform compatibility and
easy extensibility with new components.

\section{Design Decisions} \label{section:design_principles}

To satisfy the requirements stated in section \ref{section:project_outline},
we settled on a plugin based design with a small core handling the simulation main
loop and components loaded from external shared libraries.

The implementation language of choice was C++, both for the sake of familiarity
and because interfacing C code with C++ applications is extremely simple and
robust.

Qt\footnote{
%
\url{http://qt.nokia.com/products} (Qt is currently in the process of begin sold
to Digia, don't expect this link to stay valid for long.)
%
} was used as the \ac{GUI} toolkit. It is widely used, distributed and installed
by default on most Linux systems. Qt also has great cross-platform compatibility,
and as such \qsimavr should be very simple to port to any other platform that
is supported by both \simavr and Qt. The signal and slot paradigm employed by
Qt is very well suited to \ac{GUI} programming, and the
QGraphicsScene/QGraphicsView classes are very pleasant to use for more advanced
drawing and \ac{IO} tasks. Finally, QtCreator\footnote{
%
\url{http://qt.nokia.com/products/developer-tools/}
%
} is a personal favorite \ac{IDE} of
the author, which also works well with cmake.

The build system of our choice is cmake\footnote{\url{http://www.cmake.org/}}.
Again, it provides great cross-platform
compatibility and ease of use. Standard Qt specific macros (such as .ui file and
\lstinline|QObject| preprocessing handling) are available. Similar to most build tools, cmake
saves compilation time by only compiling needed (changed) files. It also handles
installation of the application after compilation. For further details, see
section \ref{section:setup_qsimavr} and the official cmake documentation at
\url{http://www.cmake.org/cmake/help/documentation.html}.

QsLog\footnote{\url{https://bitbucket.org/razvanpetru/qt-components/wiki/QsLog}}
was chosen as the logging framework for its simplicity. Currently, logging is done
to the console only, but a log file is very simple to implement.

The alternative to the chosen software ecosystem was the obvious choice of
staying close to the example component implementations included with \simavr;
namely, C code with graphical components relying on Freeglut\footnote{
%
\url{http://freeglut.sourceforge.net/}
%
}, a thin layer on top of OpenGl. However, it was clear from the start that
standard (and not-so-standard) elements such as menus, toolbars, and \ac{MDI}
windows would needed. These widgets are provided out of the box by Qt, and
would have taken an infeasable amount of time with a low level library such as
Freeglut. While the original plan was to write the \qsimavr core in C++ but keep
components in C, it turned out to feel more natural to implement entirely in C++.

Every Qt application has a main thread where the \ac{GUI} lives. The \simavr
simulation main loop runs in a separate \lstinline|QThread| and communicates
with the Qt main thread by using queued signal delivery to avoid threading
problems. The introduced latency is not detrimental to the application's purpose.

Components are implemented as plugins. They adhere to a standard interface and
can be loaded at runtime. This ensures easy extensibility; the component interface
is fixed and publicly available. New plugins can be created and loaded easily by
third parties without requiring bundled distribution with \qsimavr. Additionally,
the use of a plugin architecture forces strict separation between both
the components themselves, and between components and the core.

Each component consists of a mandatory logic and an optional \ac{GUI} part.
The logic part executes in the \simavr thread and again uses queued signals
to communicate with the \ac{GUI} part living in the main Qt thread. User input
is passed from the \ac{GUI} to the logic handler, and display information in the
other direction.

The \ac{GUI} itself needs to be able to display all component frontends, which
again can be enabled and disabled at any time. This immediately brings a multi-window
interface to mind. Since we were not completely happy with the idea of many
small windows floating individually around the desktop, we looked for further options
and soon hit on Gt's \ac{MDI} interface. The \lstinline|QMDIArea| (plus its child
components \lstinline|CMDISubWindow|) provide a multi-window interface but
encapsulate within a single ``main'' window.

It would be ideal to have flexible wiring between the components and the core.
Some of the standard BIGAVR6 components can be configured to connect to different
pins using dip-switches; the connection of external components is completely
undefined. \acp{MCU} also differ in the pins they have available. However,
component wiring in \qsimavr is currently hardcoded. This allows plugin management
to boil down to a simple on/off switch. Wirings are configured for the BIGAVR6
board used in the \ac{TU} Vienna Microcontroller course.
