%%
%% Design Approach
%%
%% This file should be edited by user
%%

\chapter{Design Approach} \label{chapter:designapproach}

It is necessary to understand simavr's internals before going on to discuss
qsimavr's design. We will begin by taking a brief look at the most important
concepts before expanding on these and further topics in simavr. Subsequently,
we will examine how qsimavr uses and expands upon simavr.

\section{simavr}

simavr is a small cross-platform AVR simulator written with simplicity and
hackability in mind. It is supported on Linux and OS X, but should run on any
platform with avr-libc support.

In the following sections, we will take a tour through simavr internals.
Without further delay, let's jump right in and walk through a short demo.

\subsection{simavr Example Walkthrough}

The following program is taken from the board\_i2ctest simavr example. Minor
modifications have been made to focus on the essential section. Error handling
is mostly omitted in favor of readability.

\begin{lstlisting}
#include <stdlib.h>
#include <stdio.h>
#include <libgen.h>
#include <pthread.h>

#include "sim_avr.h"
#include "avr_twi.h"
#include "sim_elf.h"
#include "sim_gdb.h"
#include "sim_vcd_file.h"
#include "i2c_eeprom.h"
\end{lstlisting}

The actual simulation of the external EEPROM component is located in
i2c\_eeprom.h. We will take a look at the implementation later on.

\begin{lstlisting}
avr_t * avr = NULL;
avr_vcd_t vcd_file;

i2c_eeprom_t ee;
\end{lstlisting}

avr is the main data structure. It encapsulates the entire state of the
core simulation, including register, SRAM and flash contents, the CPU state, the
current cycle count, callbacks for various tasks, pending interrupts, and more.

vcd\_file represents the file target for the \emph{value change dump} module. It
is used to dump the level changes of desired pins (or IRQ's in general) into a
file which can be subsequently viewed using utilities such as \emph{gtkwave}.

ee contains the internal state of the simulated external EEPROM.

\begin{lstlisting}
int main(int argc, char *argv[])
{
    elf_firmware_t f;
    elf_read_firmware("atmega1280_i2ctest.axf", &f);
\end{lstlisting}

The firmware is loaded from the specified file. Note that exactly the same file
can be executed on the AVR hardware without changes. MMCU and frequency
information have been embedded into the binary and are therefore available in
elf\_firmware\_t.

\begin{lstlisting}
    avr = avr_make_mcu_by_name(f.mmcu);
    avr_init(avr);
    avr_load_firmware(avr, &f);
\end{lstlisting}

The avr\_t instance is then constructed from the core file of the specified
MMCU, and initialized. The firmware is then copied into the program memory.

\begin{lstlisting}
    i2c_eeprom_init(avr, &ee, 0xa0, 0xfe, NULL, 1024);
    i2c_eeprom_attach(avr, &ee, AVR_IOCTL_TWI_GETIRQ(0));
\end{lstlisting}

AVR\_IOCTL\_TWI\_GETIRQ is a macro to retrieve the internal IRQ of the TWI
simulation. IRQ's are the main method of communication between simavr and
external components and are also used liberally throughout simavr internals.
Similar macros exist for other important AVR parts such as the ADC, IO ports,
timers, etc.

\begin{lstlisting}
    // even if not setup at startup, activate gdb if crashing
    avr->gdb_port = 1234;
    if (0) {
        //avr->state = cpu_Stopped;
        avr_gdb_init(avr);
    }
\end{lstlisting}



\begin{lstlisting}

    /*
     *  VCD file initialization
     *
     *  This will allow you to create a "wave" file and display it in gtkwave
     *  Pressing "r" and "s" during the demo will start and stop recording
     *  the pin changes
     */
//  avr_vcd_init(avr, "gtkwave_output.vcd", &vcd_file, 100000 /* usec */);
//  avr_vcd_add_signal(&vcd_file,
//      avr_io_getirq(avr, AVR_IOCTL_TWI_GETIRQ(0), TWI_IRQ_STATUS), 8 /* bits
*/ ,
//      "TWSR" );

    printf( "\nDemo launching:\n");
\end{lstlisting}



\begin{lstlisting}
    int state = cpu_Running;
    while ((state != cpu_Done) && (state != cpu_Crashed))
        state = avr_run(avr);

    return 0;
}
\end{lstlisting}


\subsection{Main Loop}
\subsection{avr\_t Initialization}
\subsection{Instruction Processing}
\subsection{Interrupts}
\subsection{Cycle Timers}
\subsection{GDB}
\subsection{IRQs}
\subsection{IO}
\subsection{Example of an internal module implementation} %TODO
\subsection{Embedding MCU Information in Binaries}
\subsection{Core Definitions}

\section{qsimavr}

Plugin based, small core, threaded, ...

%%
%% = eof =====================================================================
%%
