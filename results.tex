%%
%% Results
%%

\chapter{Results and Discussion} \label{chapter:results}

During the course if this thesis, we have developed a new open source frontend
to \simavr targeted towards simulating \ac{AVR} hardware components using a
simple, extensible, and powerful plugin architecture. Six components (simulating
eight actual hardware components) with very diverse tasks have been implemented,
serving as both initial functionality and examples for further component
implementations. Example code is available for the \ac{TWI} and 1-wire protocols.

Benefits have also reached upstream\footnote{
%
In free and open source projects, the upstream of a program or set of
programs is the project that develops those programs. [...] This term comes
from the idea that water and the goods it
carries float downstream and benefit those who are there to receive it. \cite{fedora}
In this case, \simavr is the upstream project.
%
} in the form of numerous bug fixes (especially regarding the \verb|atmega1280|
\ac{MCU}) and new features such as \ac{GDB} watchpoints. The existing documentation
has already been improved, and hopefully the \simavr chapters of this thesis
will also be of great value (sections \ref{chapter:simavr} and \ref{section:setup_simavr}).

We would like to point out that none of this would have been possible without
the open source ecosystem. We have used other developers' code, learned through
many freely available examples, and contributed back to upstream projects.
Having the source code available is also extremely important while debugging.
Of course, \qsimavr is also released as an open source project, in the hope
that it will be useful to others.

We also welcome all contributions, further developments and bug fixes. In particular
there are several topics which could use further attention and would be good
candidates for follow-up projects, both within \qsimavr and \simavr:

\begin{itemize}
\item \ac{IO} pin level handling

\simavr's current \verb|avr_ioport| implementation complicates matters when
intricate interactions with connected components are required. This issue
has already been discussed in section \ref{section:component_temperature}.
We envision a solution in which \simavr is aware of the state of the communication
partner (for example, it could be pulling the pin high with a weak pull-up resistor),
and automatically sets the pin level correctly.

\item Sleep modes

\simavr has no notion of different sleep modes. So far, we have not run into a
situation in which this turned out to be a problem, but improving simulation
accuracy is always a valid goal.

\item Component chaining

\qsimavr currently forces all components to connect directly to the core.
While we use a chaining-like mechanism with the \ac{EEPROM}, \ac{RTC}, and \ac{TWI}
components, it is currently not possible to properly connect one component to
another component.

\item Flexible wiring

At the time of writing, all wiring between components and the \ac{AVR} core is
hardcoded. Flexible wiring (allowing the user to configure which pins components
are connected to) would provide substantial advantages, both for using
components with different \ac{AVR} cores with different configurations, and
for altering the configuration of a component on the board itself (the BIGAVR6
board even allows configuring the wirings of several internal components with
dip-switches).

\item Per-component logging

Components currently print all messages to the console. It would be useful to be
able to separate logging messages by component, maybe even displaying these within
\qsimavr itself.

\item Custom component configurations

While we do store component settings, these are common to all components.
Ideally, each component should be able to manage its own custom configurations.
This could also be used for the purpose of storing permanent state such as \ac{EEPROM}
contents.

\item Cross-platform compatibility

The current version of \qsimavr has only been developed and tested on Linux.
All used technologies support several platforms, and porting \qsimavr should
take only a short time. Compilation and setup instructions would also have to be
created.
\end{itemize}
