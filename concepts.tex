%%
%% Concepts
%%

\chapter{Concepts} \label{chapter:concepts}

Simulation is the imitation of a real-world process or system over time\cite{wiki:simulation}.
There are numerous variations of this concept, including simulation for training purposes
(such as aircraft simulators), simulation of natural phenomenons for scientific study
or for creating forecast (like weather systems), and simulation simply for entertainment purposes
as can be found most prominently in computer games.

This thesis concerns itself with yet another kind of simulation, which imitates
the operation of a microcontroller and attached peripherals on common desktop computer architectures. More
specifically, \simavr simulates the family of 8-bit \ac{AVR} microcontroller systems
on any modern system supported by \emph{avr-libc}.

In this case, simulation simplifies both the development and debugging of \ac{AVR}
programs by making software engineering staples such as automatic unit tests possible
and vastly easing access to debugging methods and tools such as \ac{GDB} without further
required hardware.

Emulation is another related concept which is only indistinctly different from
simulation. Originally, emulation was first used in 1963 to refer to a type
of simulation which consisted of mixing hardware microcode and software components
to achieve faster simulation speed\cite{building_ibm}. However, this definition is now
largely obsolete. The current concensus seems to be that emulation focuses on imitating
the tangible effects of a system, while simulation concerns itself especially with
a system's inner state.

In this context, \simavr can be considered to be a cross between a simulator and
emulator; it tries to both maintain approximately realistic execution times while
producing identical execution results as when running on an \ac{AVR} \ac{MCU} as
well as maintaining and accurate internal state at all times.

Emulators usually consist of a \ac{CPU} and memory subsystems which interact with
\ac{IO} device simulations\cite{wiki:emulation}.

The \ac{CPU} subsystem is responsible
for imitating the state and operation of a processor. Specifically, essential components of an \ac{AVR}
processor are simulation of the \ac{SREG} and instruction decoding/processing.
During each simulation cycle, the subsystem decodes an instruction from program memory
and subsequently ``executes'' it by performing semantically equivalent operations on
the host \ac{CPU}. Finally, the \ac{SREG} is updated accordingly.

The memory subsystem not only keeps track of memory contents, but also limits access
to the available memory space, and handles \ac{IO} mapped registers which
trigger interactions with some \ac{IO} device when read or written.

Finally, \ac{IO} devices may be connected to the core subsystems and be programmed
to respond to (and emit) signals.

\simavr already contains the \ac{CPU} and memory subsystems, as well as some
\ac{IO} devices closely associated with \ac{AVR} \acp{MCU} such as an internal \ac{EEPROM},
\acp{ADC}, \ac{IO} ports, timers, and more. \qsimavr will extend upon this existing functionality to
implement several further \ac{IO} modules which interact seamlessly with \simavr, while also offering a
graphical user interface to control the simulation.
