%%
%%  Project:        Dissertation
%%  File:           $RCSfile$
%%  Version:        $Revision$
%%  Creation date:  Mon March 27, 2000
%%  Last changes:   $Date$
%%  Author:         $Author$
%%                  bauer@vmars.tuwien.ac.at
%%  Contents:       Main LaTeX File
%%
%%  FileID:         $Id$
%%

\documentclass[12pt,a4paper,oneside]{scrreprt}

\usepackage{graphicx}          %includegraphics-command
\usepackage{fancyheadings}
\usepackage[pdfborder={0 0 0}]{hyperref}
\usepackage[english,germanb]{babel}
\usepackage[latin1]{inputenc}  %support direct writing of German Umlauts
\usepackage{dcolumn}           %decimal column formatting
\usepackage[usenames,dvipsnames]{color}
\usepackage{listings}

\definecolor{Brown}{cmyk}{0,0.81,1,0.60}
\definecolor{OliveGreen}{cmyk}{0.64,0,0.95,0.40}
\definecolor{CadetBlue}{cmyk}{0.62,0.57,0.23,0}
\definecolor{gray}{gray}{0.5}

\lstset{
    language=C,                             % Code langugage
    basicstyle=\ttfamily,                   % Code font, Examples:
                                            % \footnotesize, \ttfamily
    keywordstyle=\color{OliveGreen},        % Keywords font ('*' = uppercase)
    commentstyle=\color{gray},              % Comments font
    captionpos=b,                           % Caption-position = bottom
    breaklines=true,                        % Automatic line breaking?
    breakatwhitespace=false,                % Automatic breaks only at
                                            % whitespace?
    showspaces=false,                       % Dont make spaces visible
    showtabs=false,                         % Dont make tabs visible
    morekeywords={__attribute__},           % Specific keywords
}

\newcolumntype{d}[1]{D{.}{.}{#1}}              % decimal formatting in tables
\newcolumntype{C}[1]{@{}>{\centering}p{#1}@{}} % centered columns with fixed width: C{width}

\usepackage{mybakktitlepage}

%%
%% ---------------------------------------------------------------------
%%

\sloppy

\oddsidemargin 1cm \evensidemargin 1cm \topmargin 0pt

\headsep 50pt \textheight 21.4cm \textwidth 14.1cm
\setlength{\parskip}{5pt plus2pt minus2pt}

\renewcommand{\floatpagefraction}{0.9}
\renewcommand{\textfraction}{0.05}
\renewcommand{\topfraction}{1.0}
\renewcommand{\bottomfraction}{1.0}

\setcounter{totalnumber}{5}
\setcounter{bottomnumber}{5}
\setcounter{topnumber}{5}

\setcounter{tocdepth}{2}
\addtolength{\abovecaptionskip}{-10pt}

\newcommand{\eg}{e.\,g., }
\newcommand{\ie}{i.\,e., }
\def\lqq{\lq\lq}
\def\rqq{\rq\rq}
\def\dq#1{\lqq #1\rqq}
\def\dqit#1{\lqq \emph{#1}\rqq}

%%
%% ---------------------------------------------------------------------
%%

%%
%% Acronym definitions
%%

\usepackage{acronym}

\newacro{ADC}{Analog-Digital Converter}
\newacro{AVR}{Alf and Vegard's Risc processor}
\newacro{CPU}{Central Processing Unit}
\newacro{EEPROM}{Electrically Erasable Programmable Read-Only Memory}
\newacro{ELF}{Executable and Linkable Format}
\newacro{FIFO}{First In, First Out}
\newacro{GDB}{GNU Debugger}
\newacro{GLCD}{Graphical Liquid Crystal Display}
\newacro{GPL}{GNU General Public License}
\newacro{GUI}{Graphical User Interface}
\newacro{ID}{Identifier}
\newacro{IDE}{Integrated Development Environment}
\newacro{IO}{Input/Output}
\newacro{IOCTL}{Input/Output Control}
\newacro{IRQ}{Interrupt Request}
\newacro{ISR}{Interrupt Service Routine}
\newacro{LCD}{Liquid Crystal Display}
\newacro{LED}{Light-Emitting Diode}
\newacro{LSB}{Least Significant Bit}
\newacro{MCU}{Microcontroller}
\newacro{MISO}{Master In, Slave Out}
\newacro{MOSI}{Master Out, Slave In}
\newacro{MSB}{Most Significant Bit}
\newacro{PC}{Program Counter}
\newacro{RTC}{Real-Time Clock}
\newacro{SP}{Stack Pointer}
\newacro{SPI}{Serial Peripheral Interface}
\newacro{SRAM}{Static Random-Access Memory}
\newacro{SREG}{Status Register}
\newacro{TWI}{Two-Wire Interface}
\newacro{UART}{Universal Asynchronous Receiver/Transmitter}
\newacro{VCD}{Value Change Dump}


%%
%% ---------------------------------------------------------------------
%%

\begin{document}

    \pagestyle{empty}
    %%
%% Title
%%
%% This file should be edited by user
%%

\title{qsimavr \\ Graphical simulation of periphery in interaction with an AVR processor}
\author{Jakob Gruber}
\address{Kirschenallee 6/1, A-2120 Obersdorf}
\matrikel{0203440}
\date{September 2012}

\advisorname{Alexander K\"ossler}
\institut{Institut f{\"u}r Technische Informatik}
\instnummer{182}

%%
%% = eof =====================================================================
%%

    \maketitle
    \cleardoublepage

    \pagestyle{plain}
    \pagenumbering{roman}
    \setlength{\parskip}{5pt plus2pt minus2pt}

    \setcounter{page}{1}

    \selectlanguage{english}
    %%
%% Abstract
%%
%% This file should be edited by user
%%

\renewcommand{\englabstractname}{}
\begin{abstract}
\begin{center}
\parbox{.95\textwidth}
{ \setlength{\parindent}{3ex}
\setlength{\parskip}{0.3\baselineskip}

\noindent Here comes the abstract. The abstract should be about
150 ... 250 words long and contain the following information:\\
{\em What is the subject of this work?}\\
{\em What is the objective of this work?}\\
{\em What were the conclusions you have drawn as a result?}\\
Remember that the abstract should be self-contained, thus does not
have any literature references in its text.
 }
\end{center}
\end{abstract}

%%
%% = eof =====================================================================
%%

    \cleardoublepage

    \setlength{\parskip}{1mm}
    \linespread{0.0}

    \tableofcontents
    \linespread{1}
    \clearpage
    \cleardoublepage
    \setlength{\parskip}{5pt plus2pt minus2pt}

    \pagestyle{fancy}
    \renewcommand{\chaptermark}[1]{\markboth{\thechapter\ #1}{}}
    \renewcommand{\sectionmark}[1]{\markright{\thesection\ #1}{}}
    \addtolength{\headheight}{2pt}

    \pagenumbering{arabic}
    \setcounter{page} {1}
    \cleardoublepage
    %%
%% Introduction
%%
%% This file should be edited by user
%%

\chapter{Introduction} \label{chapter:introduction}

The introduction should touch the following issues: \emph{General
Issue} -- what is this about, what is the field of research?
\emph{Background} -- where does this field come from, perhaps some
historic reference. What is the actual \emph{Problem Statement} you
are trying to solve in this thesis? Why is this kind of research
\emph{relevant}?

\section{Motivation and Objectives}

State the goals of your research project here. Furthermore describe
the \emph{research objectives} and the \emph{research methodology}
you are applying to solve the problem.

\section{Structure of the Thesis} \label{sec:introduction:structure}

The thesis is structured as follows:
Chapter~\ref{chapter:concepts} gives an introduction into the
basic terms and concepts used throughout the work.

...

Finally, the thesis ends with a conclusion in
Chapter~\ref{chapter:conclusion} summarizing the key results of the
presented work and giving an outlook on what can be expected from
future research in this area.

%%
%% = eof =====================================================================
%%

    \cleardoublepage
    %%
%% Concepts
%%

\chapter{Concepts} \label{chapter:concepts}

Simulation is the imitation of a real-world process or system over time\cite{wiki:simulation}.
There are numerous variations of this concept, including simulation for training purposes
(such as aircraft simulators), simulation of natural phenomenons for scientific study
or for creating forecast (like weather systems), and simulation simply for entertainment purposes
as can be found most prominently in computer games.

This thesis concerns itself with yet another kind of simulation, which imitates
the operation of a microcontroller and attached peripherals on common desktop computer architectures. More
specifically, \simavr simulates the family of 8-bit \ac{AVR} microcontroller systems
on any modern system supported by \emph{avr-libc}.

In this case, simulation simplifies both the development and debugging of \ac{AVR}
programs by making software engineering staples such as automatic unit tests possible
and vastly easing access to debugging methods and tools such as \ac{GDB} without further
required hardware.

Emulation is another related concept which is only indistinctly different from
simulation. Originally, emulation was first used in 1963 to refer to a type
of simulation which consisted of mixing hardware microcode and software components
to achieve faster simulation speed\cite{building_ibm}. However, this definition is now
largely obsolete. The current concensus seems to be that emulation focuses on imitating
the tangible effects of a system, while simulation concerns itself especially with
a system's inner state.

In this context, \simavr can be considered to be a mixture between a simulator and
emulator; it tries to both maintain approximately realistic execution times while
producing identical execution results as when running on an \ac{AVR} \ac{MCU} as
well as maintaining an accurate internal state at all times.

Emulators usually consist of a \ac{CPU} and memory subsystems which interact with
\ac{IO} device simulations\cite{wiki:emulation}.

The \ac{CPU} subsystem is responsible
for imitating the state and operation of a processor. Specifically, essential components of an \ac{AVR}
processor are simulation of the \ac{SREG} and instruction decoding/processing.
During each simulation cycle, the subsystem decodes an instruction from program memory
and subsequently ``executes'' it by performing semantically equivalent operations on
the host \ac{CPU}. Finally, the \ac{SREG} is updated accordingly.

The memory subsystem not only keeps track of memory contents, but also limits access
to the available memory space, and handles \ac{IO} mapped registers which
trigger interactions with some \ac{IO} device when read or written.

Finally, \ac{IO} devices may be connected to the core subsystems and be programmed
to respond to (and emit) signals.

\simavr already contains the \ac{CPU} and memory subsystems, as well as some
\ac{IO} devices closely associated with \ac{AVR} \acp{MCU} such as an internal \ac{EEPROM},
\acp{ADC}, \ac{IO} ports, timers, and more. \qsimavr will extend upon this existing functionality to
implement several further \ac{IO} modules which interact seamlessly with \simavr, while also offering a
graphical user interface to control the simulation.

    \cleardoublepage
    %%
%% Related Work
%%

\chapter{Related Work} \label{chapter:relatedwork}

\section{Atmel AVR Studio}
\section{simulavr}
\section{avr-sim}

% TODO

    \cleardoublepage
    %%
%% Design Approach
%%
%% This file should be edited by user
%%

\chapter{Design Approach} \label{chapter:designapproach}

Plugin based, small core, threaded, ...

%%
%% = eof =====================================================================
%%

    \cleardoublepage
    %%
%% Implementation
%%
%% This file should be edited by user
%%

\chapter{Implementation} \label{chapter:implementation}

Call this \dq{Implementation} or \dq{Case Study}, here you will
describe you actual hands-on part of your work.

\section{Types of Bachelor's Theses}

If you write a Bachelor's thesis in form of a survey, you might have
several chapters on existing work in the literature, but no chapter
as described here.


%%
%% = eof =====================================================================
%%

    \cleardoublepage
    %%
%% Results
%%
%% This file should be edited by user
%%

\chapter{Results and Discussion} \label{chapter:results}

\section{Limitation: Simplified pin level handling}

%% TODO: In simavr, a wire always has exactly one level. It is complicated
%% to simulate correct behavior when more than one level is on a wire (for example
%% a pullup and another signal.

\section{Limitation: No component chaining}

\section{Per-component logging}
\section{Per-component configuration}

%% For example: PORTD == off.

\section{Non-transient component state}

%% For example: EEPROM contents.

\section{simavr sleep modes}

%%
%% = eof =====================================================================
%%

    \cleardoublepage
    %%
%% Conclusion
%%

\chapter{Conclusion} \label{chapter:conclusion}

\qsimavr fills a gap in the open source ecosystem of \ac{AVR} utilities by
providing a simple way of simulating \ac{AVR} microcontrollers including
external periphery. The application is simple to use, and full documentation is
provided for \qsimavr and \simavr internals.

Components are implemented using a plugin architecture with a well documented
interface and numerous examples, allowing for easy extensibility by interested parties.
All components can include graphical output and accept input
from the user.

Furthermore, \ac{VCD} trace files can be produced during execution
which substantially aids development and debugging of hardware drivers.

\ac{GDB} debugging is also supported through the use of the \ac{GDB} Remote Serial
Protocol, allowing developers to use established debugging tools they are already
familiar with.

\qsimavr provides simulation for eight components that should cover
all base use cases of the BIGAVR6 board. If required, further components such
as external Bluetooth and Ethernet modules could be implemented as follow-up projects.

    \cleardoublepage
    \chapter{Acknowledgements} \label{chapter:acknowledgements}

This thesis has been made possible by the kind help of several people.

First and foremost, I'd like to thank the author of simavr, Michel Pollet, for
the help and many discussions during the past couple of months. (And of course
for writing simavr, the basis for this thesis!)

The entire open source ecosystem, for making a project like this possible in the
first place. In particular, Martin Thomas (the author of the DS18X20 demo
application), Colin O' Flynn (the author of the CRC routine used by the
temperature sensor), and Winfried Simon (for the QHexEdit widget).

My colleages Mino Sharkhawy and Ondrej Hosek for kindly providing their AVR
applications as testbeds.

Alexander K\"ossler, who has lent valuable assistance whenever I needed it;
together with the entire Microcontroller team at the University of Vienna, for
their support, code, motivation and company last semester.

And finally, Moni Linke, who has always supported me in everything I do.
    \cleardoublepage

    \appendix
    \cleardoublepage
    \addcontentsline{toc}{chapter}{Bibliography}
    \bibliography{bibfile}
    \bibliographystyle{alpha}

    %%
%% Setup Guide
%%
%% This file should be edited by user
%%

\chapter{Setup Guide}

This section provides instructions on how to retrieve, compile and install
simavr and QSimAVR on the GNU/Linux operating system.

\section{simavr}

\subsection{Getting the source code}

The official home of simavr is \url{https://github.com/buserror-uk/simavr}.
Stable releases are published as git repository tags (direct downloads are
available at \url{https://github.com/buserror-uk/simavr/tags}). To clone a local
copy of the repository, run

\begin{verbatim}
git clone git://github.com/buserror-uk/simavr.git
\end{verbatim}

\subsection{Software Dependencies}

\emph{elfutils} is the only hard dependency at run-time.

At compile-time, simavr additionally requires \emph{avr-libc} to complete its
built-in AVR core definitions. It is assumed that further standard
utilities (\emph{git}, \emph{gcc} or \emph{clang}, \emph{make}, etc \ldots) are
already present.

simavr has been tested with the following software versions:

\begin{itemize}
\item elfutils 0.154
\item avr-libc 1.8.0
\item gcc 4.7.1
\item make 3.82
\end{itemize}

\subsection{Compilation}

simavr's build system relies on standard makefiles. The simplest compilation
boils down to the usual

\begin{verbatim}
make
make install
\end{verbatim}

As usual, there are several variables to allow configuration of the build
procedure. The most important ones are described in the following section:

\begin{itemize}
\item AVR\_ROOT

The path to the system's \emph{avr-libc} installation.

While the default value
should be correct for many systems, it may need to be set manually if the
message 'WARNING \ldots did not compile, check your avr-gcc
toolchain' appears during the build. For example, if iomxx0\_1.h is located at
/usr/avr/include/avr/iomxx0\_1.h, AVR\_ROOT must be set to /usr/avr.

\item CFLAGS

The standard compiler flags variable.

It may be useful to modify CFLAGS for easier debugging (in which case
optimizations should be disabled and debugging information enabled: -O0 -g).
Additionally adding -DCONFIG\_SIMAVR\_TRACE=1 enables extra verbose output and
extended execution tracing.
\end{itemize}

These variables may be set either directly in Makefile.common, or alternatively
can be passed to the make invocation (make AVR\_ROOT=/usr/avr DESTDIR=/usr
install).

For development, we built simavr with the following procedure:

\begin{verbatim}
make clean
make AVR_ROOT=/usr/avr CFLAGS="-O0 -Wall -Wextra -g -fPIC -std=gnu99 \
  -Wno-sign-compare -Wno-unused-parameter"
make DESTDIR="/usr" install
\end{verbatim}


%%
%% = eof =====================================================================
%%

    \cleardoublepage
    %%
%% User Guide
%%
%% This file should be edited by user
%%

\chapter{User Guide}

\section{Debugging with \simavr and \ac{GDB}} \label{section:debugging}

% Mention how debugging with qsimavr is exactly the same.

%%
%% = eof =====================================================================
%%

    \cleardoublepage

\end{document}

%%
%% = eof =====================================================================
%%
