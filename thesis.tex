%%
%%  Project:        Dissertation
%%  File:           $RCSfile$
%%  Version:        $Revision$
%%  Creation date:  Mon March 27, 2000
%%  Last changes:   $Date$
%%  Author:         $Author$
%%                  bauer@vmars.tuwien.ac.at
%%  Contents:       Main LaTeX File
%%
%%  FileID:         $Id$
%%

\documentclass[12pt,a4paper,oneside]{scrreprt}

\usepackage{graphicx}          %includegraphics-command
\usepackage{fancyheadings}
\usepackage[pdfborder={0 0 0}]{hyperref}
\usepackage[english,germanb]{babel}
\usepackage[latin1]{inputenc}  %support direct writing of German Umlauts
\usepackage{dcolumn}           %decimal column formatting
\usepackage[usenames,dvipsnames]{color}
\usepackage{listings}

\definecolor{Brown}{cmyk}{0,0.81,1,0.60}
\definecolor{OliveGreen}{cmyk}{0.64,0,0.95,0.40}
\definecolor{CadetBlue}{cmyk}{0.62,0.57,0.23,0}
\definecolor{gray}{gray}{0.5}

\lstset{
    language=C,                             % Code langugage
    basicstyle=\ttfamily,                   % Code font, Examples:
                                            % \footnotesize, \ttfamily
    keywordstyle=\color{OliveGreen},        % Keywords font ('*' = uppercase)
    commentstyle=\color{gray},              % Comments font
    captionpos=b,                           % Caption-position = bottom
    breaklines=true,                        % Automatic line breaking?
    breakatwhitespace=false,                % Automatic breaks only at
                                            % whitespace?
    showspaces=false,                       % Dont make spaces visible
    showtabs=false,                         % Dont make tabs visible
    morekeywords={__attribute__},           % Specific keywords
}

\newcolumntype{d}[1]{D{.}{.}{#1}}              % decimal formatting in tables
\newcolumntype{C}[1]{@{}>{\centering}p{#1}@{}} % centered columns with fixed width: C{width}

\usepackage{mybakktitlepage}

%%
%% ---------------------------------------------------------------------
%%

\sloppy

\oddsidemargin 1cm \evensidemargin 1cm \topmargin 0pt

\headsep 50pt \textheight 21.4cm \textwidth 14.1cm
\setlength{\parskip}{5pt plus2pt minus2pt}

\renewcommand{\floatpagefraction}{0.9}
\renewcommand{\textfraction}{0.05}
\renewcommand{\topfraction}{1.0}
\renewcommand{\bottomfraction}{1.0}

\setcounter{totalnumber}{5}
\setcounter{bottomnumber}{5}
\setcounter{topnumber}{5}

\setcounter{tocdepth}{2}
\addtolength{\abovecaptionskip}{-10pt}

\newcommand{\eg}{e.\,g., }
\newcommand{\ie}{i.\,e., }
\def\lqq{\lq\lq}
\def\rqq{\rq\rq}
\def\dq#1{\lqq #1\rqq}
\def\dqit#1{\lqq \emph{#1}\rqq}

%%
%% ---------------------------------------------------------------------
%%

%%
%% Acronym definitions
%%
%% This file should be edited by user
%%

\usepackage{acronym}

% Declare acronym using:
%    \newacro{acronym}{expanded name}
% Use Acronym in text:
%    \ac{acronym}

\newacro{CPU}{Central Processing Unit}
\newacro{GPL}{GNU General Public License}
\newacro{ISR}{Interrupt Service Routine}

%%
%% ---------------------------------------------------------------------
%%

\begin{document}

    \pagestyle{empty}
    %%
%% Title
%%
%% This file should be edited by user
%%

\title{qsimavr \\ Graphical simulation of external components connected to an AVR processor}
\author{Jakob Gruber}
\address{Kirschenallee 6/1, A-2120 Obersdorf}
\matrikel{0203440}
\date{September 2012}

\advisorname{Alexander K\"ossler}
\institut{Institut f{\"u}r Technische Informatik}
\instnummer{182}

%%
%% = eof =====================================================================
%%

    \maketitle
    \cleardoublepage

    \pagestyle{plain}
    \pagenumbering{roman}
    \setlength{\parskip}{5pt plus2pt minus2pt}

    \setcounter{page}{1}

    \selectlanguage{english}
    %%
%% Abstract
%%

\renewcommand{\englabstractname}{}
\begin{abstract}
\begin{center}
\parbox{.95\textwidth}
{ \setlength{\parindent}{3ex}
\setlength{\parskip}{0.3\baselineskip}

Simulation of \acs{AVR} programs offers substantial advantages, most notably
zero costs, no hardware requirements, and easy debugging with familiar tools.
For this thesis, a new graphical, extensible \acs{AVR} simulation frontend based
on \emph{simavr} and Qt has been developed. Commonly used components such as the
\acs{GLCD}, \acs{LCD}, Temperature Sensor, \acs{EEPROM}, \acs{RTC}, \acs{LED}s
and buttons have been provided.

%TODO

}
\end{center}
\end{abstract}

%%
%% = eof =====================================================================
%%

    \cleardoublepage

    \setlength{\parskip}{1mm}
    \linespread{0.0}

    \tableofcontents
    \linespread{1}
    \clearpage
    \cleardoublepage
    \setlength{\parskip}{5pt plus2pt minus2pt}

    \pagestyle{fancy}
    \renewcommand{\chaptermark}[1]{\markboth{\thechapter\ #1}{}}
    \renewcommand{\sectionmark}[1]{\markright{\thesection\ #1}{}}
    \addtolength{\headheight}{2pt}

    \pagenumbering{arabic}
    \setcounter{page} {1}
    \cleardoublepage
    %%
%% Introduction
%%

\chapter{Introduction} \label{chapter:introduction}

Developing complex applications can be an extremely difficult task, even more so when
the targeted platform is as close to the hardware as microcontrollers. Unlike
typical computer programs, programming for a microcontroller is done at a very low level,
requiring knowledge of the exact hardware specifications and using strict protocols
to correctly communicate with external periphery. These protocols often require
sending signals with specific timing bounds (these are often in the micro- and nanosecond
range), and upon failure to do so proceed with undefined behavior instead of
printing a (hopefully) informative error message or throwing a well-defined
exception.

Debugging on the other hand is complicated by the fact that programs are executed
not on the development host, but on a separate piece of hardware.
There are several options available to developers: for example, one could
use \lstinline|printf| to output text
over a \ac{UART} interface, or toggle \acp{LED} at specific points of interest
within the program.
While debuggers may be used by connecting a \ac{JTAG} device such as the Atmel
JTAGICE mkII,
these are often expensive\footnote{
%
And specifically, not available to students in the \ac{TU} Vienna Microcontroller course.
%
} and can introduce side effects of their own.

The answer to these problems lies in simulation. Binary \ac{ELF} files can be
interpreted by a simulator to mimic the state and actions of an \ac{AVR} controller
executing an \ac{ELF} firmware file. This program can not only run directly on the development
host, but also implement sophisticated development aids such as debugging support
using standard tools, producing execution and signal traces, and replaying recorded
signal sequences. Simulation simplifies development of \ac{AVR} programmers and improves
the quality of produced artifacts while simultaneously lowering the entry barrier
for new programmers by allowing them to debug without acquiring expensive
hardware (or even doing initial development without possessing a hardware microcontroller
at all). Unit testing and regression checks can be created that run automatically
on the development machine, and environments may be simulated by recording and then
replaying signal sequences at will.

The aim of this thesis is to produce such a tool for \ac{AVR} microntrollers,
specifically the \verb|atmega1280| processor running on a frequency of 16 \ac{MHz}
on a MikroElektronika BIGAVR6 Development System\footnote{
%
\url{http://www.mikroe.com/bigavr/}, accessed 2012-09-01.
%
}, based on the \ac{AVR} core simulation provided by \simavr\footnote{
%
\url{https://github.com/buserror-uk/simavr}, accessed 2012-09-01.
%
}. External periphery of the BIGAVR6 board will also be simulated, including
the \ac{LCD} and \ac{GLCD} displays, the \ac{TWI} \ac{EEPROM} and \ac{RTC},
on-board \acp{LED} and push buttons connected to the \ac{IO} ports, a touchscreen
attached to the \ac{GLCD}, and a temperature sensor using the 1-wire protocol.

This document is structured as follows: Chapter \ref{chapter:introduction} briefly
introduces the topic of this work and its objectives.

Chapter \ref{chapter:concepts} includes a short summary of the concept of simulations,
and computer simulations in particular.

Chapter \ref{chapter:relatedwork} summarizes the current state of related \ac{AVR}
simulation software.

The internals of \simavr are examined in-depth in chapter \ref{chapter:simavr}.

Chapter \ref{chapter:designapproach} discusses the project requirements, and
how the chosen design approach will achieve these as well as important decisions
made during the development process.

Chapter \ref{chapter:implementation} walks through the implementation of the
\qsimavr core as well as the provided components.

Results and potential for further work are presented in chapter \ref{chapter:results}.

Chapter \ref{chapter:conclusion} contains the conclusion of this thesis.

Appendix \ref{chapter:setup} provides a setup guide for getting up and running
with both \simavr and \qsimavr.

Finally, appendix \ref{chapter:user_guide} contains a \qsimavr user guide
and brief tutorial for debugging \ac{AVR} programs with \qsimavr and \ac{GDB}.

    \cleardoublepage
    %%
%% Concepts
%%
%% This file should be edited by user
%%

\chapter{Concepts} \label{chapter:concepts}

Usually, this chapter is necessary in order to give an overview on
the \emph{terms} and \emph{concepts} that are required to understand
your work.

\section{Writing Style}

Usually you should not use the first person singular (\emph{I}) in
your text, write \emph{we} instead. As a general recommendation, use
the first person sparsely, sometimes it can be replaced by a phrase
like \emph{This work presents...}.

The indefinite article \textbf{a} is used as \textbf{an} before a
vowel sound - for example \textbf{an} apple, \textbf{an} hour,
\textbf{an} unusual thing, \textbf{an} FPGA (becourse the acronym is
pronouned Ef-Pee-Gee-A), \textbf{an} HIL. Before a consonant sound
represented by a vowel letter \textbf{a} is usual -- for example
\textbf{a} one, \textbf{a} unique thing, \textbf{a} historic
chance\footnote{According to Merriam Webster, both \textbf{a} and
\textbf{an} can be used in writing before unstressed or weakly
stressed syllables with initial h, thus you could also write
\textbf{an} historic chance.}.

\section{Acronyms}

Explain acronyms at their first occurrence in the text. In order to
achieve this consistently, we recommend to use the \texttt{acronym}
package.

A new acronym is then declared by writing
\verb+\newacro{acronym}{expanded name}+. Use the macro
\verb+\ac{acronym}+ as a placeholder for the acronym in the text.

See file \texttt{acronym.tex} for further examples and explanations.

\newacro{GPL}{Gnu Public License}

\section{Figures}

A Figure should always be referenced in the text, as it is the case
with Figure~\ref{fig:example}.

\begin{figure}[h]
 \centerline{\includegraphics[width=.5\columnwidth]{pics/example}}
  \caption{Example figure}
  \label{fig:example}
\end{figure}

This template can be compiled with the \texttt{latex} command or the
\texttt{pdflatex} command. While \texttt{latex} creates an
intermediate file format (.dvi) that can be further processed into a
\texttt{.ps} or \texttt{.pdf} file, the \texttt{pdflatex} command
directly creates a \texttt{.pdf} file.

Note that with \texttt{latex} the \verb+\includegraphics+ accepts
only .eps files, while with \texttt{pdflatex} accepts \texttt{.pdf},
\texttt{.png}, or \texttt{.jpg}. Luckily, the file extension can be
omitted in order that \verb+\includegraphics{pics/example}+ will
look for file with name \texttt{example.eps} in \texttt{latex} mode
and for a file with name \texttt{example.pdf}, \texttt{example.png},
or \texttt{example.jpg} in \texttt{pdflatex} mode. If you already
have an \texttt{.eps} file, you may create a respective
\texttt{.pdf} file with the commandline conversion tool
\texttt{epstopdf}.

\section{Citations and References}

Whenever you refer to previously published work, you should set a
reference to acknowledge the work you build upon. For example this
is a reference to two bachelor's theses~\cite{kraut:2003,
weirich:2005}. If you literally cite a part of someone else's work,
mark the respective sentence by quotes and italic letters and add
the page number, where is text can be found:

\dqit{An intelligent or {\em smart} transducer is the integration of
an analog or digital sensor or actuator element, a processing unit,
and a communication interface. In case of a sensor, the smart
transducer transforms the raw sensor signal to a standardized
digital representation, checks and calibrates the signal, and
transmits this digital signal to its users via a standardized
communication protocol.}\cite[p.\,175]{elmenreich:2005}

\section{Spellchecking}

Do not use your advisor as your spell checker. Instead, run an
electronic spell checker over your document before submitting the
document to your advisor.

\section{References with Bibtex}

Bibtex is an additional program to {\LaTeX} that creates a list of
your cited references in a chapter named {\em Bibliography}. In
order to use Bibtex, you must maintain a database of all references
in so-called \emph{bibfiles} (file extension \texttt{.bib}).

The \emph{bibfiles} contain entries of several types, the most
needed types are \texttt{book}, \texttt{inproceedings},
\texttt{article}, \texttt{techreport}, \texttt{mastersthesis}, and
\texttt{phdthesis}. In the following we list the templates for these
types, whereas each asterisk (*) should be replaced by the
respective data, if this is not available, the element should be
left out. The case of the element names does not matter to Bibtex,
however in the examples we have used UPPERCASE for the obligatory
fields and lowercase for the optional fields. To see some complete
examples, have a look into the file \texttt{bibfile.bib}. For more
information, read~\cite{patashnik:1988}.

\subsection{Some BibteX Examples}

\footnotesize
\begin{verbatim}
@BOOK{*,
  AUTHOR =       {*},
  editor =       {*},
  TITLE =        {*},
  PUBLISHER =    {*},
  YEAR =         {*},
  volume =       {*},
  number =       {*},
  series =       {*},
  address =      {*},
  edition =      {*},
  month =        {*},
  note =         {*}
}

@INPROCEEDINGS{*,
  AUTHOR =       {*},
  TITLE =        {*},
  BOOKTITLE =    {*},
  YEAR =         {*},
  editor =       {*},
  volume =       {*},
  number =       {*},
  series =       {*},
  pages =        {*},
  address =      {*},
  month =        {*},
  organization = {*},
  publisher =    {*},
  note =         {*}
}

@ARTICLE{*,
  AUTHOR =       {*},
  TITLE =        {*},
  JOURNAL =      {*},
  YEAR =         {*},
  volume =       {*},
  number =       {*},
  pages =        {*},
  month =        {*},
  note =         {*}
}

@TECHREPORT{*,
  AUTHOR =       {*},
  TITLE =        {*},
  INSTITUTION =  {*},
  YEAR =         {*},
  type =         {*},
  number =       {*},
  address =      {*},
  month =        {*},
  note =         {*}
}

@MASTERSTHESIS{*,
  AUTHOR =       {*},
  TITLE =        {*},
  SCHOOL =       {*},
  YEAR =         {*},
  type =         {*},
  address =      {*},
  month =        {*},
  note =         {*}
}

@PHDTHESIS{*,
  AUTHOR =       {*},
  TITLE =        {*},
  SCHOOL =       {*},
  YEAR =         {*},
  type =         {*},
  address =      {*},
  month =        {*},
  note =         {*},
  abstract =     {*},
  keywords =     {*},
  source =       {*},
} \end{verbatim}




%%
%% = eof =====================================================================
%%

    \cleardoublepage
    %%
%% Related Work
%%
%% This file should be edited by user
%%

\chapter{Related Work} \label{chapter:relatedwork}

This chapter should give an overview over existing work that is
related to your work. Instead of \dq{Related Work}, this chapter can
also be named specifically to the topic of the thesis.

For example in Bernhard Weirich's Bachelor's thesis, there are two
chapters on related work named \dq{Time-Driven Algorithms} and
\dq{Event-Driven Algorithms}~\cite{weirich:2005}.

Each related approach should be described by a section of about
100-500 words.

\section{Types of Bachelor's Theses}

If you write a plain report on some implementation, you might have
no chapter on related works.

%%
%% = eof =====================================================================
%%

    \cleardoublepage
    %%
%% Design Approach
%%
%% This file should be edited by user
%%

%%%%%%%%%%%%%%%%%%%%%%%%%%%%%%%%%%%%%%%%%%%%%%%%%%%%%%%%%%%%%%%%%%%%%%%%%%%%%%%%
\chapter{simavr Internals} \label{chapter:simavr}
%%%%%%%%%%%%%%%%%%%%%%%%%%%%%%%%%%%%%%%%%%%%%%%%%%%%%%%%%%%%%%%%%%%%%%%%%%%%%%%%

It is necessary to understand simavr's internals before going on to discuss
qsimavr's design.

simavr is a small cross-platform \ac{AVR} simulator written with simplicity and
hackability in mind. It is supported on Linux and OS X, but should run on any
platform with avr-libc support.

In the following sections, we will take a tour through simavr internals\footnote{
Most, if not all of the code examined in this chapter is taken directly from simavr.}.
Without further delay, let's jump right in and walk through a short demo.


%%%%%%%%%%%%%%%%%%%%%%%%%%%%%%%%%%%%%%%%%%%%%%%%%%%%%%%%%%%%%%%%%%%%%%%%%%%%%%%%
\section{simavr Example Walkthrough}
%%%%%%%%%%%%%%%%%%%%%%%%%%%%%%%%%%%%%%%%%%%%%%%%%%%%%%%%%%%%%%%%%%%%%%%%%%%%%%%%

The following program is taken from the board\_i2ctest simavr example. Minor
modifications have been made to focus on the essential section. Error handling
is mostly omitted in favor of readability.

\begin{lstlisting}
#include <stdlib.h>
#include <stdio.h>
#include <libgen.h>
#include <pthread.h>

#include "sim_avr.h"
#include "avr_twi.h"
#include "sim_elf.h"
#include "sim_gdb.h"
#include "sim_vcd_file.h"
#include "i2c_eeprom.h"
\end{lstlisting}

The actual simulation of the external \ac{EEPROM} component is located in
i2c\_eeprom.h. We will take a look at the implementation later on.

\begin{lstlisting}
avr_t * avr = NULL;
avr_vcd_t vcd_file;

i2c_eeprom_t ee;
\end{lstlisting}

avr is the main data structure. It encapsulates the entire state of the
core simulation, including register, \ac{SRAM} and flash contents, the \ac{CPU} state, the
current cycle count, callbacks for various tasks, pending interrupts, and more.

\lstinline|vcd_file| represents the file target for the \emph{value change dump} module. It
is used to dump the level changes of desired pins (or \ac{IRQ}'s in general) into a
file which can be subsequently viewed using utilities such as \emph{gtkwave}.

\lstinline|ee| contains the internal state of the simulated external \ac{EEPROM}.

\begin{lstlisting}
int main(int argc, char *argv[])
{
    elf_firmware_t f;
    elf_read_firmware("atmega1280_i2ctest.axf", &f);
\end{lstlisting}

The firmware is loaded from the specified file. Note that exactly the same file
can be executed on the \ac{AVR} hardware without changes. \ac{MCU} and frequency
information have been embedded into the binary and are therefore available in
\lstinline|elf_firmware_t|.

\begin{lstlisting}
    avr = avr_make_mcu_by_name(f.mmcu);
    avr_init(avr);
    avr_load_firmware(avr, &f);
\end{lstlisting}

The \lstinline|avr_t| instance is then constructed from the core file of the
specified \ac{MCU} and initialized. \lstinline|avr_load_firmware| copies the
firmware into program memory.

\begin{lstlisting}
    i2c_eeprom_init(avr, &ee, 0xa0, 0xfe, NULL, 1024);
    i2c_eeprom_attach(avr, &ee, AVR_IOCTL_TWI_GETIRQ(0));
\end{lstlisting}

\lstinline|AVR_IOCTL_TWI_GETIRQ| is a macro to retrieve the internal \ac{IRQ} of the \ac{TWI}
simulation. \ac{IRQ}'s are the main method of communication between simavr and
external components and are also used liberally throughout simavr internals.
Similar macros exist for other important \ac{AVR} parts such as the \ac{ADC}, \ac{IO} ports,
timers, etc.

\begin{lstlisting}
    avr->gdb_port = 1234;
    avr->state = cpu_Stopped;
    avr_gdb_init(avr);
\end{lstlisting}

This section sets up simavr's \ac{GDB} infrastructure to listen on port 1234. The
\ac{CPU} is stopped to allow \ac{GDB} to attach before execution begins.

\begin{lstlisting}
    avr_vcd_init(avr, "gtkwave_output.vcd", &vcd_file, 100000 /* usec */);
    avr_vcd_add_signal(
        &vcd_file,
        avr_io_getirq(avr, AVR_IOCTL_TWI_GETIRQ(0), TWI_IRQ_STATUS),
        8 /* bits */,
        "TWSR");
\end{lstlisting}

Next, a value change dump output is configured to track changes to the
\lstinline|TWI_IRQ_STATUS| \ac{IRQ}. The file may then be viewed using the \emph{gtkwave}
application.

\begin{lstlisting}
    int state = cpu_Running;
    while ((state != cpu_Done) && (state != cpu_Crashed))
        state = avr_run(avr);

    return 0;
}
\end{lstlisting}

Finally, we have reached the simple main loop. Each iteration executes one
instruction, handles any pending interrupts and cycle timers, and sleeps if
possible. As soon as execution completes or crashes, simulation stops and we
exit the program.

We will now examine the relevant parts of the \lstinline|i2c_eeprom| implementation.
Details have been omitted and only communication with the \lstinline|avr_t| instance are
shown.

\begin{lstlisting}
static const char * _ee_irq_names[2] = {
		[TWI_IRQ_MISO] = "8>eeprom.out",
		[TWI_IRQ_MOSI] = "32<eeprom.in",
};

void
i2c_eeprom_init(
		struct avr_t * avr,
		i2c_eeprom_t * p,
		uint8_t addr,
		uint8_t mask,
		uint8_t * data,
		size_t size)
{

    /* [...] */

	p->irq = avr_alloc_irq(&avr->irq_pool, 0, 2, _ee_irq_names);
	avr_irq_register_notify(p->irq + TWI_IRQ_MOSI, i2c_eeprom_in_hook, p);

    /* [...] */
}
\end{lstlisting}

First, the \ac{EEPROM} allocates its own private \ac{IRQ}s. The \ac{EEPROM} implementation
does not know or care to which simavr \ac{IRQ}'s they will be attached. It then
attaches a callback function (\lstinline|i2c_eeprom_in_hook|) to the \ac{MOSI} \ac{IRQ}. This
function will be called whenever a value is written to the \ac{IRQ}. The pointer to
the \ac{EEPROM} state p is passed to each of these callback function calls.

\begin{lstlisting}
void
i2c_eeprom_attach(
		struct avr_t * avr,
		i2c_eeprom_t * p,
		uint32_t i2c_irq_base )
{
	avr_connect_irq(
		p->irq + TWI_IRQ_MISO,
		avr_io_getirq(avr, i2c_irq_base, TWI_IRQ_MISO));
	avr_connect_irq(
		avr_io_getirq(avr, i2c_irq_base, TWI_IRQ_MOSI),
		p->irq + TWI_IRQ_MOSI );
}
\end{lstlisting}

The private \ac{IRQ}s are then attached to simavr's internal \ac{IRQ}s. This is called
chaining - all messages raised are forwarded to all chained \ac{IRQ}s.

\begin{lstlisting}
static void
i2c_eeprom_in_hook(
		struct avr_irq_t * irq,
		uint32_t value,
		void * param)
{
	i2c_eeprom_t * p = (i2c_eeprom_t*)param;

    /* [...] */

    avr_raise_irq(p->irq + TWI_IRQ_MISO,
            avr_twi_irq_msg(TWI_COND_ACK, p->selected, 1));

    /* [...] */
}
\end{lstlisting}

Finally, we've reached the \ac{IRQ} callback function. It is responsible for
simulating communications between simavr (acting as the \ac{TWI} master) and the
\ac{EEPROM} (as the \ac{TWI} slave). The \ac{EEPROM} state which was previously passed to
\lstinline|avr_irq_register_notify| is contained in the \lstinline|param| variable and cast back to
an \lstinline|i2c_eeprom_t| pointer for further use.

Outgoing messages are sent by raising the internal \ac{IRQ}. This message is then
forwarded to all chained \ac{IRQ}s.


%%%%%%%%%%%%%%%%%%%%%%%%%%%%%%%%%%%%%%%%%%%%%%%%%%%%%%%%%%%%%%%%%%%%%%%%%%%%%%%%
\section{Main Loop} \label{Main Loop}
%%%%%%%%%%%%%%%%%%%%%%%%%%%%%%%%%%%%%%%%%%%%%%%%%%%%%%%%%%%%%%%%%%%%%%%%%%%%%%%%

We will now take a closer look at the main loop implementation. Each call to
\lstinline|avr_run| triggers the function stored in the run member of the \lstinline|avr_t| structure
(\lstinline|avr->run|\footnote{Whenever \lstinline|avr| is mentioned in a code
section, it is assumed to be the main \lstinline|avr_t| struct.}).
The two standard implementations are \lstinline|avr_callback_run_raw| and
\lstinline|avr_callback_run_gdb|, located in sim\_avr.c. The essence of both function is
identical; since \lstinline|avr_callback_run_gdb| contains additional logic for \ac{GDB}
handling (network protocol, stepping), we will examine it further and point out
any differences to the the raw version. Several comments and irrelevant code
sections have been removed.

\begin{lstlisting}
void avr_callback_run_gdb(avr_t * avr)
{
    avr_gdb_processor(avr, avr->state == cpu_Stopped);

    if (avr->state == cpu_Stopped)
        return ;

    int step = avr->state == cpu_Step;
    if (step)
        avr->state = cpu_Running;
\end{lstlisting}

This initial section is \ac{GDB} specific. \lstinline|avr_gdb_processor| is responsible for
handling \ac{GDB} network communication. It also checks if execution has reached a
breakpoint or the end of a step and stops the \ac{CPU} if it did.

If \ac{GDB} has transmitted a step command, we need to save the state during the
main section of the loop (the \ac{CPU} ``runs'' for one instruction) and restore to
the ``StepDone'' state at on completion.

\begin{lstlisting}
    avr_flashaddr_t new_pc = avr->pc;

    if (avr->state == cpu_Running) {
        new_pc = avr_run_one(avr);
    }
\end{lstlisting}

We have now reached the actual execution of the current instruction. If the \ac{CPU}
is currently running, \lstinline|avr_run_one| decodes the instruction located in flash memory
(\lstinline|avr->flash|) and triggers all necessary actions. This can include setting the \ac{CPU}
state (SLEEP), updating the status register \ac{SREG}, writing or reading from memory
locations, altering the program counter PC, etc \ldots

Finally, the cycle counter (\lstinline|avr->cycle|) is updated and the new
program counter is returned.

\begin{lstlisting}
    if (avr->sreg[S_I] && !avr->i_shadow)
        avr->interrupts.pending_wait++;
    avr->i_shadow = avr->sreg[S_I];
\end{lstlisting}

This section ensures that interrupts are not triggered immediately when
enabling the interrupt flag in the status register, but with an (additional)
delay of one instruction.

\begin{lstlisting}
    avr_cycle_count_t sleep = avr_cycle_timer_process(avr);
    avr->pc = new_pc;
\end{lstlisting}

Next, all due cycle timers are processed. Cycle timers are one of the
most important and heavily used mechanisms in simavr. A timer allows scheduling
execution of a callback function once a specific count of execution cycles have
passed, thus simulating events which occur after a specific amount of time has
passed. For example, the \lstinline|avr_timer| module uses cycle timers to schedule timer
interrupts.

The returned estimated sleep time is set to the next pending event cycle (or a
hardcoded limit of 1000 cycles if none exist).

\begin{lstlisting}
    if (avr->state == cpu_Sleeping) {
        if (!avr->sreg[S_I]) {
            avr->state = cpu_Done;
            return;
        }
        avr->sleep(avr, sleep);
        avr->cycle += 1 + sleep;
    }
\end{lstlisting}

If the \ac{CPU} is currently sleeping, the time spent is simulated using the callback
stored in \lstinline|avr->sleep|. In \ac{GDB} mode, the time is used to listen for
\ac{GDB} commands, while the raw version simply calls usleep.

It is worth noting that
we have improved the timing behavior by accumulating requested sleep cycles until
a minimum of 200 usec has been reached. usleep cannot handle lower sleep times
accurately, which caused an unrealistic execution slowdown.

A special case occurs when the \ac{CPU} is sleeping while interrupts are turned off.
In this scenario, there is way of ever waking up. Therefore, execution is halted
gracefully.

\begin{lstlisting}
    if (avr->state == cpu_Running || avr->state == cpu_Sleeping)
        avr_service_interrupts(avr);
\end{lstlisting}

Finally, any immediately pending interrupts are handled. The highest priority
interrupt (this depends solely on the interrupt vector address) is removed from
the pending queue, interrupts are disabled in the status register, and the
program counter is set to the interrupt vector.

If the \ac{CPU} is sleeping, interrupts can be raised by cycle timers.

\begin{lstlisting}
    if (step)
        avr->state = cpu_StepDone;
}
\end{lstlisting}

Wrapping up, if the current loop iteration was a \ac{GDB} step, the state is set
such that the next iteration will inform \ac{GDB} and halt the \ac{CPU}.


%%%%%%%%%%%%%%%%%%%%%%%%%%%%%%%%%%%%%%%%%%%%%%%%%%%%%%%%%%%%%%%%%%%%%%%%%%%%%%%%
\section{Initialization}
%%%%%%%%%%%%%%%%%%%%%%%%%%%%%%%%%%%%%%%%%%%%%%%%%%%%%%%%%%%%%%%%%%%%%%%%%%%%%%%%

%%%%%%%%%%%%%%%%%%%%%%%%%%%%%%%%%%%%%%%%%%%%%%%%%%%%%%%%%%%%%%%%%%%%%%%%%%%%%%%%
\subsection{\lstinline|avr_t| Initialization}
%%%%%%%%%%%%%%%%%%%%%%%%%%%%%%%%%%%%%%%%%%%%%%%%%%%%%%%%%%%%%%%%%%%%%%%%%%%%%%%%

The \lstinline|avr_t| struct requires some somple initialization before it is
ready to be used by the main loop as discussed in section \ref{Main Loop}.

\lstinline|avr_make_mcu_by_name| fills in all details specific to an \ac{MCU}. This
includes settings such as memory sizes, register locations, available components,
the default \ac{CPU} frequency, etc \ldots

The \ac{MCU} definitions are located in the simavr/cores subdirectory of the simavr
source tree and are compiled conditionally depending on the the local avr-libc
support. A complete list of locally supported cores is printed by running simavr
without any arguments.

On successful completion, it returns a pointer to the \lstinline|avr_t| struct.

If \ac{GDB} support is desired, \lstinline|avr->gdb_port| must be set, and
\lstinline|avr_gdb_init| must be called to create the required data structures,
set the \lstinline|avr->run| and \lstinline|avr->sleep| callbacks, and listen
on the specified port. It is also recommended to initially stop the cpu
(\lstinline|avr->state = cpu_Stopped|) to delay program execution until it
is started manually by \ac{GDB}.

Further settings can now be applied manually (typical candidates are logging and
tracing levels).

%%%%%%%%%%%%%%%%%%%%%%%%%%%%%%%%%%%%%%%%%%%%%%%%%%%%%%%%%%%%%%%%%%%%%%%%%%%%%%%%
\subsection{Firmware}
%%%%%%%%%%%%%%%%%%%%%%%%%%%%%%%%%%%%%%%%%%%%%%%%%%%%%%%%%%%%%%%%%%%%%%%%%%%%%%%%

We now have a fully initialized \lstinline|avr_t| struct and are ready to load
code. This is accomplished using \lstinline|avr_read_firmware|, which uses
elfutils to decode the \ac{ELF} file and read it into an \lstinline|elf_firmware_t|
struct and \lstinline|avr_load_firmware| to load its contents into the
\lstinline|avr_t| struct.

Besides loading the program code into \lstinline|avr->flash| (and \ac{EEPROM} contents
into \lstinline|avr->eeprom|, if available), there are several useful extended
features which can be embedded directly into the \ac{ELF} file.

The target \ac{MCU}, frequency and voltages can be specified in the \ac{ELF} file by using the
\lstinline|AVR_MCU| and \lstinline|AVR_MCU_VOLTAGES| macros provided by
avr\_mcu\_section.h:

\begin{lstlisting}
#include "avr_mcu_section.h"
AVR_MCU(16000000 /* Hz */, "atmega1280");
AVR_MCU_VOLTAGES(3300 /* milliVolt */, 3300 /* milliVolt */, 3300 /* milliVolt */);
\end{lstlisting}

\ac{VCD} traces can be set up automatically. The following code will create an 8-bit
trace on the UDR0 register, and a trace masked to display only the UDRE0 bit of
the UCSR0A register.

\begin{lstlisting}
const struct avr_mmcu_vcd_trace_t _mytrace[]  _MMCU_ = {
    { AVR_MCU_VCD_SYMBOL("UDR0"), .what = (void*)&UDR0, },
    { AVR_MCU_VCD_SYMBOL("UDRE0"), .mask = (1 << UDRE0), .what = (void*)&UCSR0A, },
};
\end{lstlisting}

Several predefined commands can be sent from the firmware to simavr during program execution.
At the time of writing, these include starting and stopping \ac{VCD} traces, and putting
UART0 into loopback mode. An otherwise unused register must be specified
to listen for command requests. During execution, writing a command to this
register will trigger the associated action within simavr.

\begin{lstlisting}
AVR_MCU_SIMAVR_COMMAND(&GPIOR0);

int main() {
    /* [...] */
    GPIOR0 = SIMAVR_CMD_\ac{VCD}_START_TRACE;
    /* [...] */
}
\end{lstlisting}

Likewise, a register can be specified for use as a debugging output. All bytes
written to this register will be output to the console.

\begin{lstlisting}
AVR_MCU_SIMAVR_CONSOLE(&GPIOR0);

int main() {
    /* [...] */
    const char *s = "Hello World\r";
    for (const char *t = s; *t; t++)
        GPIOR0 = *t;
    /* [...] */
}
\end{lstlisting}

Usually, UART0 is used for this purpose. The simplest debug output can be achieved
by binding \lstinline|stdout| to \lstinline|UART0| as described by the avr-libc
documentation \cite{libc}, and then using \lstinline|printf| and similar functions.
This alternate console output is provided in case using UART0 is not possible or desired.



%%%%%%%%%%%%%%%%%%%%%%%%%%%%%%%%%%%%%%%%%%%%%%%%%%%%%%%%%%%%%%%%%%%%%%%%%%%%%%%%
\section{Instruction Processing}
%%%%%%%%%%%%%%%%%%%%%%%%%%%%%%%%%%%%%%%%%%%%%%%%%%%%%%%%%%%%%%%%%%%%%%%%%%%%%%%%

We have now covered \lstinline|avr_t| initialization, the main loop, and loading
firmware files. But how are instructions actually decoded and executed? Let's
take a look at \lstinline|avr_run_one|, located in sim\_core.

The opcode is reconstructed by retrieving the two bytes located at
\lstinline|avr->flash[avr->pc]|. \lstinline|avr->pc| points to the \ac{LSB}, and
\lstinline|avr->pc + 1| to the \ac{MSB}. Thus, the full opcode is reconstructed with:

\begin{lstlisting}
uint32_t opcode = (avr->flash[avr->pc + 1] << 8) | avr->flash[avr->pc];
\end{lstlisting}

As we have seen, \lstinline|avr->pc| represents the byte address in flash memory.
Therefore, the next instruction is located at \lstinline|avr->pc + 2|. This
default new program counter may still be altered in the course of processing
in case of jumps, branches, calls and larger opcodes such as STS\cite{instructionset}.

Note also that the \ac{AVR} flash addresses are usually represented as word addresses
(\lstinline|avr->pc >> 1|).

Similar to the program counter, the spent cycles are set to a default value of 1.

The instruction and its operands are then extracted from the opcode and processed
in a large switch statement. The instructions themselves can be roughly categorized
into arithmetic and logic instructions, branch instructions, data transfer
instructions, bit and bit-test instructions and \ac{MCU} control instructions.

Processing these will involve a number of typical tasks:

\begin{itemize}
\item Status register modifications

The status register is stored in \lstinline|avr->sreg| as a byte array.
Most instructions alter the \ac{SREG} in some way, and convenience functions such as
\lstinline|get_compare_carry| are used to ease this task. Note that whenever the
firmware reads from \ac{SREG}, it must be reconstructed from \lstinline|avr->sreg|.

\item Reading or writing memory

\lstinline|_avr_set_ram| is used to write bytes to a specific address. Accessing
an \ac{SREG} will trigger a reconstruction similar to what has been discussed above.
\ac{IO} register accesses trigger any connected \ac{IO} callbacks and raise all associated
\ac{IRQ}s. If a \ac{GDB} watchpoint has been hit, the \ac{CPU} is stopped and a status report
is sent to \ac{GDB}. Data watchpoint support has been added by the author.

\item Modifying the program counter

Jumps, skips, calls, returns and similar instructions alter the program counter.
This is achieved by simply setting \lstinline|new_pc| to an appropriate value. Care must be
taken to skip 32 bit instructions correctly.

\item Altering \ac{MCU} state

Instructions such as SLEEP and BREAK directly alter the state of the simulation.

\item Stack operations

Pushing and popping the stack involve altering the stack pointer in addition
to the actual memory access. 
\end{itemize}

Upon conclusion, \lstinline|avr->cycle| is updated with the actual instruction
duration, and the new program counter is returned.

%%%%%%%%%%%%%%%%%%%%%%%%%%%%%%%%%%%%%%%%%%%%%%%%%%%%%%%%%%%%%%%%%%%%%%%%%%%%%%%%
\section{Interrupts}
%%%%%%%%%%%%%%%%%%%%%%%%%%%%%%%%%%%%%%%%%%%%%%%%%%%%%%%%%%%%%%%%%%%%%%%%%%%%%%%%

An interrupt is an asynchronous signal which causes the the \ac{CPU} to jump to
the associated \ac{ISR} and continue execution there. In the \ac{AVR} architecture,
the interrupt priority is ordered according to its place in the interrupt
vector table. When an interrupt is serviced, interrupts are disabled globally.

%%%%%%%%%%%%%%%%%%%%%%%%%%%%%%%%%%%%%%%%%%%%%%%%%%%%%%%%%%%%%%%%%%%%%%%%%%%%%%%%
\subsection{Data Structures}
%%%%%%%%%%%%%%%%%%%%%%%%%%%%%%%%%%%%%%%%%%%%%%%%%%%%%%%%%%%%%%%%%%%%%%%%%%%%%%%%

Let's take a look at how interrupts are represented in simavr:

\begin{lstlisting}
typedef struct avr_int_vector_t {
    uint8_t         vector;
    avr_regbit_t    enable;
    avr_regbit_t    raised;
    avr_irq_t       irq;
    uint8_t         pending : 1,
                    trace : 1,
                    raise_sticky : 1;
} avr_int_vector_t;
\end{lstlisting}

Each interrupt vector has an \lstinline|avr_int_vector_t|. \lstinline|vector| is
actual vector address, for example \lstinline|INT0_vect|. \lstinline|enable|
and \lstinline|raised| specify the \ac{IO} register index for, respectively, the
interrupt enable flag and the interrupt raised bit (again taking \lstinline|INT0|
as an example, enable would point to the \lstinline|INT0| bit in \lstinline|EIMSK|,
and raised to \lstinline|INTF0| in \lstinline|EIFR|. \lstinline|irq| is raised to
1 when the interrupt is triggered, and to 0 when it is serviced.

Usually, raised flags are cleared automatically upon interrupt servicing. However,
this does not count for all interrupts(notably, \lstinline|TWINT|).
\lstinline|raise_sticky| was introduced by the author to handle this special case.

Interrupt vector definitions are stored in an \lstinline|avr_int_table_t|,
\lstinline|avr->interrupts|.

\begin{lstlisting}
typedef struct  avr_int_table_t {
    avr_int_vector_t * vector[64];
    uint8_t         vector_count;
    uint8_t         pending_wait;
    avr_int_vector_t * pending[64];
    uint8_t         pending_w,
                    pending_r;
} avr_int_table_t, *avr_int_table_p;
\end{lstlisting}

\lstinline|pending_wait| stores the number of cycles to wait before servicing
pending interrupts. This simulates the real interrupt delay that occurs between
raising and servicing, and whenever interrupts are enabled
(and previously disabled).

\lstinline|pending| along with \lstinline|pending_w| and \lstinline|pending_r|
represents a ringbuffer of pending interrupts. Note that servicing an
interrupt removes the one with the highest priority.


%%%%%%%%%%%%%%%%%%%%%%%%%%%%%%%%%%%%%%%%%%%%%%%%%%%%%%%%%%%%%%%%%%%%%%%%%%%%%%%%
\subsection{Raising and Servicing Interrupts}
%%%%%%%%%%%%%%%%%%%%%%%%%%%%%%%%%%%%%%%%%%%%%%%%%%%%%%%%%%%%%%%%%%%%%%%%%%%%%%%%

When an interrupt \lstinline|vector| is raised, \lstinline|vector->pending| is
set, \lstinline|vector| is added to the \lstinline|pending| \ac{FIFO} of
\lstinline|avr->interrupts|, and a non-zero \lstinline|pending_wait| time is
ensured. If the \ac{CPU} is currently sleeping, it is woken up.

As we've already covered in section \ref{Main Loop}, servicing interrupts is
only attempted if the \ac{CPU} is either running or sleeping. Additionally,
interrupts must be enabled globally in \ac{SREG}, and \lstinline|pending_wait|
(which is decremented on each \lstinline|avr_service_interrupts| call) must have
reached zero. The next pending vector with highest priority is then removed from
the pending ringbuffer and serviced as follows:

\begin{lstlisting}
if (!avr_regbit_get(avr, vector->enable) || !vector->pending) {
    vector->pending = 0;
\end{lstlisting}

If the specific interrupt is masked or has been cleared, no action occurs.

\begin{lstlisting}
} else {
    _avr_push16(avr, avr->pc >> 1);
    avr->sreg[S_I] = 0;
    avr->pc = vector->vector * avr->vector_size;
    avr_clear_interrupt(avr, vector);
}
\end{lstlisting}

Otherwise, the current program counter is pushed onto the stack. This illustrates
the difference between byte addresses (as used in \lstinline|avr->pc|) and
word addresses (as expected by the \ac{AVR} processor).
Interrupts are then disabled by clearing the I bit of the status register, and
the program counter is set to the \ac{ISR} vector. Finally, if
\lstinline|raise_sticky| is 0, the interrupt flag is cleared.

%%%%%%%%%%%%%%%%%%%%%%%%%%%%%%%%%%%%%%%%%%%%%%%%%%%%%%%%%%%%%%%%%%%%%%%%%%%%%%%%
\section{Cycle Timers}
%%%%%%%%%%%%%%%%%%%%%%%%%%%%%%%%%%%%%%%%%%%%%%%%%%%%%%%%%%%%%%%%%%%%%%%%%%%%%%%%

Cycle timers allow scheduling an event after a certain amount of cycles have
passed.

\begin{lstlisting}
typedef avr_cycle_count_t (*avr_cycle_timer_t)(
        struct avr_t * avr,
        avr_cycle_count_t when,
        void * param);

void
avr_cycle_timer_register(
        struct avr_t * avr,
        avr_cycle_count_t when,
        avr_cycle_timer_t timer,
        void * param);
\end{lstlisting}

\lstinline|when| is the minimum count of cycles that must pass until
\lstinline|timer| is called (\lstinline|param| and \lstinline|when| are passed
back to \lstinline|timer|\footnote{
QSimAVR exploits \lstinline|param| to implement
callbacks to class instances by passing the \lstinline|this| pointer as
\lstinline|param|.}

Once dispatched, the cycle timer is removed from the list of pending timers. If
it returns a nonzero value, it is readded to occur at or after that cycle has
been reached. It is important to realize that it therefore differs from the
\lstinline|when| argument of \lstinline|avr_cycle_timer_register|, which expects
a relative cycle count (in contrast to the absolute cycle count returned by the
callback itself).

The cycle timer system is used during the main loop to determine sleep durations;
if there are any pending timers, the sleep callback may sleep until the next timer
is scheduled. Otherwise, a default value of 1000 cycles is returned. \ac{IRQ}s
and interrupts caused by external events (for example, a ``touch'' event transmitted
from the simulated touchscreen component) are and can \emph{not} be taken into
account.

This means that scheduled sleep times will always be simulated to completion by
\lstinline|avr->sleep|, even if an external event causing \ac{CPU} wakeup is
triggered immediately after going to sleep.
Given a situation in which the next scheduled timer is many cycles in the future
and the \ac{CPU} is currently sleeping, the simulation will become extremely
unresponsive to external events.

However, in real applications this situation is very unprobable, since
manual events (which cannot be scheduled through cycle timers) occur very rarely,
and most applications will have at least some cycle timers with a short period.

It is worth remembering though, that cycle timers are the preferred and most
accurate method of scheduling interrupts in simavr.


%%%%%%%%%%%%%%%%%%%%%%%%%%%%%%%%%%%%%%%%%%%%%%%%%%%%%%%%%%%%%%%%%%%%%%%%%%%%%%%%
\section{\acf{GDB}}
%%%%%%%%%%%%%%%%%%%%%%%%%%%%%%%%%%%%%%%%%%%%%%%%%%%%%%%%%%%%%%%%%%%%%%%%%%%%%%%%

%%%%%%%%%%%%%%%%%%%%%%%%%%%%%%%%%%%%%%%%%%%%%%%%%%%%%%%%%%%%%%%%%%%%%%%%%%%%%%%%
\section{\acf{IRQ}}
%%%%%%%%%%%%%%%%%%%%%%%%%%%%%%%%%%%%%%%%%%%%%%%%%%%%%%%%%%%%%%%%%%%%%%%%%%%%%%%%

%%%%%%%%%%%%%%%%%%%%%%%%%%%%%%%%%%%%%%%%%%%%%%%%%%%%%%%%%%%%%%%%%%%%%%%%%%%%%%%%
\section{\acf{IO}}
%%%%%%%%%%%%%%%%%%%%%%%%%%%%%%%%%%%%%%%%%%%%%%%%%%%%%%%%%%%%%%%%%%%%%%%%%%%%%%%%

%%%%%%%%%%%%%%%%%%%%%%%%%%%%%%%%%%%%%%%%%%%%%%%%%%%%%%%%%%%%%%%%%%%%%%%%%%%%%%%%
\section{\acf{VCD}}
%%%%%%%%%%%%%%%%%%%%%%%%%%%%%%%%%%%%%%%%%%%%%%%%%%%%%%%%%%%%%%%%%%%%%%%%%%%%%%%%

%%%%%%%%%%%%%%%%%%%%%%%%%%%%%%%%%%%%%%%%%%%%%%%%%%%%%%%%%%%%%%%%%%%%%%%%%%%%%%%%
\section{Example of an internal module implementation} %TODO
%%%%%%%%%%%%%%%%%%%%%%%%%%%%%%%%%%%%%%%%%%%%%%%%%%%%%%%%%%%%%%%%%%%%%%%%%%%%%%%%

%%%%%%%%%%%%%%%%%%%%%%%%%%%%%%%%%%%%%%%%%%%%%%%%%%%%%%%%%%%%%%%%%%%%%%%%%%%%%%%%
\section{Embedding \acs{MCU} Information in Binaries}
%%%%%%%%%%%%%%%%%%%%%%%%%%%%%%%%%%%%%%%%%%%%%%%%%%%%%%%%%%%%%%%%%%%%%%%%%%%%%%%%

%%%%%%%%%%%%%%%%%%%%%%%%%%%%%%%%%%%%%%%%%%%%%%%%%%%%%%%%%%%%%%%%%%%%%%%%%%%%%%%%
\section{Core Definitions}
%%%%%%%%%%%%%%%%%%%%%%%%%%%%%%%%%%%%%%%%%%%%%%%%%%%%%%%%%%%%%%%%%%%%%%%%%%%%%%%%

%%
%% = eof =====================================================================
%%

    \cleardoublepage
    %%
%% Design Approach
%%
%% This file should be edited by user
%%

\chapter{Design Approach} \label{chapter:designapproach}

It is necessary to understand simavr's internals before going on to discuss
qsimavr's design. We will begin by taking a brief look at the most important
concepts before expanding on these and further topics in simavr. Subsequently,
we will examine how qsimavr uses and expands upon simavr.

\section{simavr}

simavr is a small cross-platform AVR simulator written with simplicity and
hackability in mind. It is supported on Linux and OS X, but should run on any
platform with avr-libc support.

In the following sections, we will take a tour through simavr internals.
Without further delay, let's jump right in and walk through a short demo.

\subsection{simavr Example Walkthrough}

The following program is taken from the board\_i2ctest simavr example. Minor
modifications have been made to focus on the essential section. Error handling
is mostly omitted in favor of readability.

\begin{lstlisting}
#include <stdlib.h>
#include <stdio.h>
#include <libgen.h>
#include <pthread.h>

#include "sim_avr.h"
#include "avr_twi.h"
#include "sim_elf.h"
#include "sim_gdb.h"
#include "sim_vcd_file.h"
#include "i2c_eeprom.h"
\end{lstlisting}

The actual simulation of the external EEPROM component is located in
i2c\_eeprom.h. We will take a look at the implementation later on.

\begin{lstlisting}
avr_t * avr = NULL;
avr_vcd_t vcd_file;

i2c_eeprom_t ee;
\end{lstlisting}

avr is the main data structure. It encapsulates the entire state of the
core simulation, including register, SRAM and flash contents, the CPU state, the
current cycle count, callbacks for various tasks, pending interrupts, and more.

vcd\_file represents the file target for the \emph{value change dump} module. It
is used to dump the level changes of desired pins (or IRQ's in general) into a
file which can be subsequently viewed using utilities such as \emph{gtkwave}.

ee contains the internal state of the simulated external EEPROM.

\begin{lstlisting}
int main(int argc, char *argv[])
{
    elf_firmware_t f;
    elf_read_firmware("atmega1280_i2ctest.axf", &f);
\end{lstlisting}

The firmware is loaded from the specified file. Note that exactly the same file
can be executed on the AVR hardware without changes. MMCU and frequency
information have been embedded into the binary and are therefore available in
elf\_firmware\_t.

\begin{lstlisting}
    avr = avr_make_mcu_by_name(f.mmcu);
    avr_init(avr);
    avr_load_firmware(avr, &f);
\end{lstlisting}

The avr\_t instance is then constructed from the core file of the specified
MMCU, and initialized. The firmware is then copied into the program memory.

\begin{lstlisting}
    i2c_eeprom_init(avr, &ee, 0xa0, 0xfe, NULL, 1024);
    i2c_eeprom_attach(avr, &ee, AVR_IOCTL_TWI_GETIRQ(0));
\end{lstlisting}

AVR\_IOCTL\_TWI\_GETIRQ is a macro to retrieve the internal IRQ of the TWI
simulation. IRQ's are the main method of communication between simavr and
external components and are also used liberally throughout simavr internals.
Similar macros exist for other important AVR parts such as the ADC, IO ports,
timers, etc.

\begin{lstlisting}
    // even if not setup at startup, activate gdb if crashing
    avr->gdb_port = 1234;
    if (0) {
        //avr->state = cpu_Stopped;
        avr_gdb_init(avr);
    }
\end{lstlisting}



\begin{lstlisting}

    /*
     *  VCD file initialization
     *
     *  This will allow you to create a "wave" file and display it in gtkwave
     *  Pressing "r" and "s" during the demo will start and stop recording
     *  the pin changes
     */
//  avr_vcd_init(avr, "gtkwave_output.vcd", &vcd_file, 100000 /* usec */);
//  avr_vcd_add_signal(&vcd_file,
//      avr_io_getirq(avr, AVR_IOCTL_TWI_GETIRQ(0), TWI_IRQ_STATUS), 8 /* bits
*/ ,
//      "TWSR" );

    printf( "\nDemo launching:\n");
\end{lstlisting}



\begin{lstlisting}
    int state = cpu_Running;
    while ((state != cpu_Done) && (state != cpu_Crashed))
        state = avr_run(avr);

    return 0;
}
\end{lstlisting}


\subsection{Main Loop}
\subsection{avr\_t Initialization}
\subsection{Instruction Processing}
\subsection{Interrupts}
\subsection{Cycle Timers}
\subsection{GDB}
\subsection{IRQs}
\subsection{IO}
\subsection{Example of an internal module implementation} %TODO
\subsection{Embedding MCU Information in Binaries}
\subsection{Core Definitions}

\section{qsimavr}

Plugin based, small core, threaded, ...

%%
%% = eof =====================================================================
%%

    \cleardoublepage
    %%
%% Implementation
%%
%% This file should be edited by user
%%

\chapter{Implementation} \label{chapter:implementation}

Call this \dq{Implementation} or \dq{Case Study}, here you will
describe you actual hands-on part of your work.

\section{Types of Bachelor's Theses}

If you write a Bachelor's thesis in form of a survey, you might have
several chapters on existing work in the literature, but no chapter
as described here.


%%
%% = eof =====================================================================
%%

    \cleardoublepage
    %%
%% Results
%%

\chapter{Results and Discussion} \label{chapter:results}

During the course if this thesis, we have developed a new open source frontend
to \simavr targeted towards simulating \ac{AVR} hardware components using a
simple, extensible, and powerful plugin architecture. Six components (simulating
eight actual hardware components) with very diverse tasks have been implemented,
serving as both initial functionality and examples for further component
implementations. Example code is available for the \ac{TWI} and 1-wire protocols.

Benefits have also reached upstream\footnote{
%
In free and open source projects, the upstream of a program or set of
programs is the project that develops those programs. Fedora is downstream
of those projects. This term comes from the idea that water and the goods it
carries float downstream and benefit those who are there to receive it. \cite{fedora}
In this case, \simavr is the upstream project.
%
} in the form of numerous bug fixes (especially regarding the \verb|atmega1280|
\ac{MCU}) and new features such as \ac{GDB} watchpoints. The existing documentation
has already been improved, and hopefully the \simavr chapters of this thesis
will also be of great value (sections \ref{chapter:simavr} and \ref{section:setup_simavr}).

We would like to point out that none of this would have been possible without
the open source ecosystem. We have used other developers' code, learned through
many freely available examples, and contributed back to upstream projects.
Having the source code available is also extremely important while debugging.
Of course, \qsimavr is also released as an open source project, in the hope
that it will be useful to others.

We also welcome all contributions, further developments and bug fixes. In particular
there are several topics which could use further attention and would be good
candidates for follow-up projects, both within \qsimavr and \simavr:

\begin{itemize}
\item \ac{IO} pin level handling

\simavr's current \verb|avr_ioport| implementation complicates matters when
intricate interactions with connected components are required. This issue
has already been discussed in section \ref{section:component_temperature}.
We envision a solution in which \simavr is aware of the state of the communication
partner (for example, it could be pulling the pin high with a weak pull-up resistor),
and automatically sets the pin level correctly.

\item Sleep modes

\simavr has no notion of different sleep modes. So far, we have not run into a
situation in which this turned out to be a problem, but improving simulation
accuracy is always a valid goal.

\item Component chaining

\qsimavr currently forces all components to connect directly to the core.
While we use a chaining-like mechanism with the \ac{EEPROM}, \ac{RTC}, and \ac{TWI}
components, it is currently not possible to properly connect one component to
another component.

\item Flexible wiring

At the time of writing, all wiring between components and the \ac{AVR} core is
hardcoded. Flexible wiring (allowing the user to configure which pins components
are connected to) would provide substantial advantages, both for using
components with different \ac{AVR} cores with different configurations, and
for altering the configuration of a component on the board itself (the BIGAVR6
board even allows configuring the wirings of several internal components with
dip-switches).

\item Per-component logging

Components currently print all messages to the console. It would be useful to be
able to separate logging messages by component, maybe even displaying these within
\qsimavr itself.

\item Custom component configurations

While we do store component settings, these are common to all components.
Ideally, each component should be able to manage its own custom configurations.
This could also be used for the purpose of storing permanent state such as \ac{EEPROM}
contents.

\item Cross-platform compatibility

The current version of \qsimavr has only been developed and tested on Linux.
All used technologies support several platforms, and porting \qsimavr should
take only a short time. Compilation and setup instructions would also have to be
created.
\end{itemize}

    \cleardoublepage
    %%
%% Conclusion
%%
%% This file should be edited by user
%%

\chapter{Conclusion} \label{chapter:conclusion}


%%
%% = eof =====================================================================
%%

    \cleardoublepage
    \chapter{Acknowledgements} \label{chapter:acknowledgements}

This thesis has been made possible by the kind help of several people.

First and foremost, I'd like to thank the author of simavr, Michel Pollet, for
the help and many discussions during the past couple of months. (And of course
for writing simavr, the basis for this thesis!)

The entire open source ecosystem, for making a project like this possible in the
first place. In particular, Martin Thomas (the author of the DS18X20 demo
application), Colin O' Flynn (the author of the CRC routine used by the
temperature sensor), and Winfried Simon (for the QHexEdit widget).

My colleages Mino Sharkhawy and Ondrej Hosek for kindly providing their AVR
applications as testbeds.

Alexander K\"ossler, who has lent valuable assistance whenever I needed it;
together with the entire Microcontroller team at the University of Vienna, for
their support, code, motivation and company last semester.

And finally, Moni Linke, who has always supported me in everything I do.
    \cleardoublepage

    \appendix
    \cleardoublepage
    \addcontentsline{toc}{chapter}{Bibliography}
    \bibliography{bibfile}
    \bibliographystyle{alpha}
    \nocite{*}

    %%
%% Setup Guide
%%
%% This file should be edited by user
%%

\chapter{Setup Guide}

This section provides instructions on how to retrieve, compile and install
simavr and QSimAVR on the GNU/Linux operating system.

\section{simavr}

\subsection{Getting the source code}

The official home of simavr is \url{https://github.com/buserror-uk/simavr}.
Stable releases are published as git repository tags (direct downloads are
available at \url{https://github.com/buserror-uk/simavr/tags}). To clone a local
copy of the repository, run

\begin{verbatim}
git clone git://github.com/buserror-uk/simavr.git
\end{verbatim}

\subsection{Software Dependencies}

\emph{elfutils} is the only hard dependency at run-time.

At compile-time, simavr additionally requires \emph{avr-libc} to complete its
built-in AVR core definitions. It is assumed that further standard
utilities (\emph{git}, \emph{gcc} or \emph{clang}, \emph{make}, etc \ldots) are
already present.

simavr has been tested with the following software versions:

\begin{itemize}
\item elfutils 0.154
\item avr-libc 1.8.0
\item gcc 4.7.1
\item make 3.82
\end{itemize}

\subsection{Compilation}

simavr's build system relies on standard makefiles. The simplest compilation
boils down to the usual

\begin{verbatim}
make
make install
\end{verbatim}

As usual, there are several variables to allow configuration of the build
procedure. The most important ones are described in the following section:

\begin{itemize}
\item AVR\_ROOT

The path to the system's \emph{avr-libc} installation.

While the default value
should be correct for many systems, it may need to be set manually if the
message 'WARNING \ldots did not compile, check your avr-gcc
toolchain' appears during the build. For example, if iomxx0\_1.h is located at
/usr/avr/include/avr/iomxx0\_1.h, AVR\_ROOT must be set to /usr/avr.

\item CFLAGS

The standard compiler flags variable.

It may be useful to modify CFLAGS for easier debugging (in which case
optimizations should be disabled and debugging information enabled: -O0 -g).
Additionally adding -DCONFIG\_SIMAVR\_TRACE=1 enables extra verbose output and
extended execution tracing.
\end{itemize}

These variables may be set either directly in Makefile.common, or alternatively
can be passed to the make invocation (make AVR\_ROOT=/usr/avr DESTDIR=/usr
install).

For development, we built simavr with the following procedure:

\begin{verbatim}
make clean
make AVR_ROOT=/usr/avr CFLAGS="-O0 -Wall -Wextra -g -fPIC -std=gnu99 \
  -Wno-sign-compare -Wno-unused-parameter"
make DESTDIR="/usr" install
\end{verbatim}


%%
%% = eof =====================================================================
%%

    \cleardoublepage
    %%
%% User Guide
%%
%% This file should be edited by user
%%

\chapter{User Guide}

\section{Debugging with \simavr and \ac{GDB}} \label{section:debugging}

% Mention how debugging with qsimavr is exactly the same.

%%
%% = eof =====================================================================
%%

    \cleardoublepage

\end{document}

%%
%% = eof =====================================================================
%%
