%%
%% Design Approach
%%
%% This file should be edited by user
%%

%%%%%%%%%%%%%%%%%%%%%%%%%%%%%%%%%%%%%%%%%%%%%%%%%%%%%%%%%%%%%%%%%%%%%%%%%%%%%%%%
\chapter{simavr Internals} \label{chapter:simavr}
%%%%%%%%%%%%%%%%%%%%%%%%%%%%%%%%%%%%%%%%%%%%%%%%%%%%%%%%%%%%%%%%%%%%%%%%%%%%%%%%

It is necessary to understand simavr's internals before going on to discuss
qsimavr's design.

simavr is a small cross-platform AVR simulator written with simplicity and
hackability in mind. It is supported on Linux and OS X, but should run on any
platform with avr-libc support.

In the following sections, we will take a tour through simavr internals.
Without further delay, let's jump right in and walk through a short demo.


%%%%%%%%%%%%%%%%%%%%%%%%%%%%%%%%%%%%%%%%%%%%%%%%%%%%%%%%%%%%%%%%%%%%%%%%%%%%%%%%
\section{simavr Example Walkthrough}
%%%%%%%%%%%%%%%%%%%%%%%%%%%%%%%%%%%%%%%%%%%%%%%%%%%%%%%%%%%%%%%%%%%%%%%%%%%%%%%%

The following program is taken from the board\_i2ctest simavr example. Minor
modifications have been made to focus on the essential section. Error handling
is mostly omitted in favor of readability.

\begin{lstlisting}
#include <stdlib.h>
#include <stdio.h>
#include <libgen.h>
#include <pthread.h>

#include "sim_avr.h"
#include "avr_twi.h"
#include "sim_elf.h"
#include "sim_gdb.h"
#include "sim_vcd_file.h"
#include "i2c_eeprom.h"
\end{lstlisting}

The actual simulation of the external EEPROM component is located in
i2c\_eeprom.h. We will take a look at the implementation later on.

\begin{lstlisting}
avr_t * avr = NULL;
avr_vcd_t vcd_file;

i2c_eeprom_t ee;
\end{lstlisting}

avr is the main data structure. It encapsulates the entire state of the
core simulation, including register, SRAM and flash contents, the CPU state, the
current cycle count, callbacks for various tasks, pending interrupts, and more.

\lstinline|vcd_file| represents the file target for the \emph{value change dump} module. It
is used to dump the level changes of desired pins (or IRQ's in general) into a
file which can be subsequently viewed using utilities such as \emph{gtkwave}.

ee contains the internal state of the simulated external EEPROM.

\begin{lstlisting}
int main(int argc, char *argv[])
{
    elf_firmware_t f;
    elf_read_firmware("atmega1280_i2ctest.axf", &f);
\end{lstlisting}

The firmware is loaded from the specified file. Note that exactly the same file
can be executed on the AVR hardware without changes. MMCU and frequency
information have been embedded into the binary and are therefore available in
\lstinline|elf_firmware_t|.

\begin{lstlisting}
    avr = avr_make_mcu_by_name(f.mmcu);
    avr_init(avr);
    avr_load_firmware(avr, &f);
\end{lstlisting}

The \lstinline|avr_t| instance is then constructed from the core file of the specified
MMCU, and initialized. The firmware is then copied into the program memory.

\begin{lstlisting}
    i2c_eeprom_init(avr, &ee, 0xa0, 0xfe, NULL, 1024);
    i2c_eeprom_attach(avr, &ee, AVR_IOCTL_TWI_GETIRQ(0));
\end{lstlisting}

\lstinline|AVR_IOCTL_TWI_GETIRQ| is a macro to retrieve the internal IRQ of the TWI
simulation. IRQ's are the main method of communication between simavr and
external components and are also used liberally throughout simavr internals.
Similar macros exist for other important AVR parts such as the ADC, IO ports,
timers, etc.

\begin{lstlisting}
    avr->gdb_port = 1234;
    avr->state = cpu_Stopped;
    avr_gdb_init(avr);
\end{lstlisting}

This section sets up simavr's gdb infrastructure to listen on port 1234. The
CPU is stopped to allow gdb to attach before execution begins.

\begin{lstlisting}
    avr_vcd_init(avr, "gtkwave_output.vcd", &vcd_file, 100000 /* usec */);
    avr_vcd_add_signal(
        &vcd_file,
        avr_io_getirq(avr, AVR_IOCTL_TWI_GETIRQ(0), TWI_IRQ_STATUS),
        8 /* bits */,
        "TWSR");
\end{lstlisting}

Next, a value change dump output is configured to track changes to the
\lstinline|TWI_IRQ_STATUS| IRQ. The file may then be viewed using the \emph{gtkwave}
application.

\begin{lstlisting}
    int state = cpu_Running;
    while ((state != cpu_Done) && (state != cpu_Crashed))
        state = avr_run(avr);

    return 0;
}
\end{lstlisting}

Finally, we have reached the simple main loop. Each iteration executes one
instruction, handles any pending interrupts and cycle timers, and sleeps if
possible. As soon as execution completes or crashes, simulation stops and we
exit the program.

We will now examine the relevant parts of the \lstinline|i2c_eeprom| implementation.
Details have been omitted and only communication with the \lstinline|avr_t| instance are
shown.

\begin{lstlisting}
static const char * _ee_irq_names[2] = {
		[TWI_IRQ_MISO] = "8>eeprom.out",
		[TWI_IRQ_MOSI] = "32<eeprom.in",
};

void
i2c_eeprom_init(
		struct avr_t * avr,
		i2c_eeprom_t * p,
		uint8_t addr,
		uint8_t mask,
		uint8_t * data,
		size_t size)
{

    /* [...] */

	p->irq = avr_alloc_irq(&avr->irq_pool, 0, 2, _ee_irq_names);
	avr_irq_register_notify(p->irq + TWI_IRQ_MOSI, i2c_eeprom_in_hook, p);

    /* [...] */
}
\end{lstlisting}

First, the EEPROM allocates its own private IRQs. The EEPROM implementation
does not know or care to which simavr IRQ's they will be attached. It then
attaches a callback function (\lstinline|i2c_eeprom_in_hook|) to the MOSI IRQ. This
function will be called whenever a value is written to the IRQ. The pointer to
the EEPROM state p is passed to each of these callback function calls.

\begin{lstlisting}
void
i2c_eeprom_attach(
		struct avr_t * avr,
		i2c_eeprom_t * p,
		uint32_t i2c_irq_base )
{
	avr_connect_irq(
		p->irq + TWI_IRQ_MISO,
		avr_io_getirq(avr, i2c_irq_base, TWI_IRQ_MISO));
	avr_connect_irq(
		avr_io_getirq(avr, i2c_irq_base, TWI_IRQ_MOSI),
		p->irq + TWI_IRQ_MOSI );
}
\end{lstlisting}

The private IRQs are then attached to simavr's internal IRQs. This is called
chaining - all messages raised are forwarded to all chained IRQs.

\begin{lstlisting}
static void
i2c_eeprom_in_hook(
		struct avr_irq_t * irq,
		uint32_t value,
		void * param)
{
	i2c_eeprom_t * p = (i2c_eeprom_t*)param;

    /* [...] */

    avr_raise_irq(p->irq + TWI_IRQ_MISO,
            avr_twi_irq_msg(TWI_COND_ACK, p->selected, 1));

    /* [...] */
}
\end{lstlisting}

Finally, we've reached the IRQ callback function. It is responsible for
simulating communications between simavr (acting as the TWI master) and the
EEPROM (as the TWI slave). The EEPROM state which was previously passed to
\lstinline|avr_irq_register_notify| is contained in the param variable and cast back to
an \lstinline|i2c_eeprom_t| pointer for further use.

Outgoing messages are sent by raising the internal IRQ. This message is then
forwarded to all chained IRQs.


%%%%%%%%%%%%%%%%%%%%%%%%%%%%%%%%%%%%%%%%%%%%%%%%%%%%%%%%%%%%%%%%%%%%%%%%%%%%%%%%
\section{Main Loop}
%%%%%%%%%%%%%%%%%%%%%%%%%%%%%%%%%%%%%%%%%%%%%%%%%%%%%%%%%%%%%%%%%%%%%%%%%%%%%%%%

We will now take a closer look at the main loop implementation. Each call to
\lstinline|avr_run| triggers the function stored in the run member of the \lstinline|avr_t| structure
(\lstinline|avr->run|). The two standard implementations are \lstinline|avr_callback_run_raw| and
\lstinline|avr_callback_run_gdb|, located in sim\_avr.c. The essence of both function is
identical; since \lstinline|avr_callback_run_gdb| contains additional logic for GDB
handling (network protocol, stepping), we will examine it further and point out
any differences to the the raw version. Several comments and irrelevant code
sections have been removed.

\begin{lstlisting}
void avr_callback_run_gdb(avr_t * avr)
{
    avr_gdb_processor(avr, avr->state == cpu_Stopped);

    if (avr->state == cpu_Stopped)
        return ;

    int step = avr->state == cpu_Step;
    if (step)
        avr->state = cpu_Running;
\end{lstlisting}

This initial section is gdb specific. \lstinline|avr_gdb_processor| is responsible for
handling GDB network communication. It also checks if execution has reached a
breakpoint or the end of a step and stops the CPU if it did.

If GDB has transmitted a step command, we need to save the state during the
main section of the loop (the CPU ``runs'' for one instruction) and restore to
the ``StepDone'' state at on completion.

\begin{lstlisting}
    avr_flashaddr_t new_pc = avr->pc;

    if (avr->state == cpu_Running) {
        new_pc = avr_run_one(avr);
    }
\end{lstlisting}

We have now reached the actual execution of the current instruction. If the CPU
is currently running, \lstinline|avr_run_one| decodes the instruction located in flash memory
(\lstinline|avr->flash|) and triggers all necessary actions. This can include setting the CPU
state (SLEEP), updating the status register SREG, writing or reading from memory
locations, altering the program counter PC, etc \ldots

Finally, the cycle counter (\lstinline|avr->cycle|) is updated and the new
program counter is returned.

\begin{lstlisting}
    if (avr->sreg[S_I] && !avr->i_shadow)
        avr->interrupts.pending_wait++;
    avr->i_shadow = avr->sreg[S_I];
\end{lstlisting}

This section ensures that interrupts are not triggered immediately when
enabling the interrupt flag in the status register, but with an (additional)
delay of one instruction.

\begin{lstlisting}
    avr_cycle_count_t sleep = avr_cycle_timer_process(avr);
    avr->pc = new_pc;
\end{lstlisting}

Next, all due cycle timers are processed. Cycle timers are one of the
most important and heavily used mechanisms in simavr. A timer allows scheduling
execution of a callback function once a specific count of execution cycles have
passed, thus simulating events which occur after a specific amount of time has
passed. For example, the \lstinline|avr_timer| module uses cycle timers to schedule timer
interrupts.

The returned estimated sleep time is set to the next pending event cycle (or a
hardcoded limit of 1000 cycles if none exist).

\begin{lstlisting}
    if (avr->state == cpu_Sleeping) {
        if (!avr->sreg[S_I]) {
            avr->state = cpu_Done;
            return;
        }
        avr->sleep(avr, sleep);
        avr->cycle += 1 + sleep;
    }
\end{lstlisting}

If the CPU is currently sleeping, the time spent is simulated using the callback
stored in \lstinline|avr->sleep|. In GDB mode, the time is used to listen for
GDB commands, while the raw version simply calls usleep.

It is worth noting that
we have improved the timing behavior by accumulating requested sleep cycles until
a minimum of 200 usec has been reached. usleep cannot handle lower sleep times
accurately, which caused an unrealistic execution slowdown.

A special case occurs when the CPU is sleeping while interrupts are turned off.
In this scenario, there is way of ever waking up. Therefore, execution is halted
gracefully.

\begin{lstlisting}
    if (avr->state == cpu_Running || avr->state == cpu_Sleeping)
        avr_service_interrupts(avr);
\end{lstlisting}

Finally, any immediately pending interrupts are handled. The highest priority
interrupt (this depends solely on the interrupt vector address) is removed from
the pending queue, interrupts are disabled in the status register, and the
program counter is set to the interrupt vector.

If the CPU is sleeping, interrupts can be raised by cycle timers.

\begin{lstlisting}
    if (step)
        avr->state = cpu_StepDone;
}
\end{lstlisting}

Wrapping up, if the current loop iteration was a GDB step, the state is set
such that the next iteration will inform GDB and halt the CPU.


%%%%%%%%%%%%%%%%%%%%%%%%%%%%%%%%%%%%%%%%%%%%%%%%%%%%%%%%%%%%%%%%%%%%%%%%%%%%%%%%
\section{\lstinline|avr_t| Initialization}
%%%%%%%%%%%%%%%%%%%%%%%%%%%%%%%%%%%%%%%%%%%%%%%%%%%%%%%%%%%%%%%%%%%%%%%%%%%%%%%%

%%%%%%%%%%%%%%%%%%%%%%%%%%%%%%%%%%%%%%%%%%%%%%%%%%%%%%%%%%%%%%%%%%%%%%%%%%%%%%%%
\section{Instruction Processing}
%%%%%%%%%%%%%%%%%%%%%%%%%%%%%%%%%%%%%%%%%%%%%%%%%%%%%%%%%%%%%%%%%%%%%%%%%%%%%%%%

%%%%%%%%%%%%%%%%%%%%%%%%%%%%%%%%%%%%%%%%%%%%%%%%%%%%%%%%%%%%%%%%%%%%%%%%%%%%%%%%
\section{Interrupts}
%%%%%%%%%%%%%%%%%%%%%%%%%%%%%%%%%%%%%%%%%%%%%%%%%%%%%%%%%%%%%%%%%%%%%%%%%%%%%%%%

%%%%%%%%%%%%%%%%%%%%%%%%%%%%%%%%%%%%%%%%%%%%%%%%%%%%%%%%%%%%%%%%%%%%%%%%%%%%%%%%
\section{Cycle Timers}
%%%%%%%%%%%%%%%%%%%%%%%%%%%%%%%%%%%%%%%%%%%%%%%%%%%%%%%%%%%%%%%%%%%%%%%%%%%%%%%%

%%%%%%%%%%%%%%%%%%%%%%%%%%%%%%%%%%%%%%%%%%%%%%%%%%%%%%%%%%%%%%%%%%%%%%%%%%%%%%%%
\section{GDB}
%%%%%%%%%%%%%%%%%%%%%%%%%%%%%%%%%%%%%%%%%%%%%%%%%%%%%%%%%%%%%%%%%%%%%%%%%%%%%%%%

%%%%%%%%%%%%%%%%%%%%%%%%%%%%%%%%%%%%%%%%%%%%%%%%%%%%%%%%%%%%%%%%%%%%%%%%%%%%%%%%
\section{IRQs}
%%%%%%%%%%%%%%%%%%%%%%%%%%%%%%%%%%%%%%%%%%%%%%%%%%%%%%%%%%%%%%%%%%%%%%%%%%%%%%%%

%%%%%%%%%%%%%%%%%%%%%%%%%%%%%%%%%%%%%%%%%%%%%%%%%%%%%%%%%%%%%%%%%%%%%%%%%%%%%%%%
\section{IO}
%%%%%%%%%%%%%%%%%%%%%%%%%%%%%%%%%%%%%%%%%%%%%%%%%%%%%%%%%%%%%%%%%%%%%%%%%%%%%%%%

%%%%%%%%%%%%%%%%%%%%%%%%%%%%%%%%%%%%%%%%%%%%%%%%%%%%%%%%%%%%%%%%%%%%%%%%%%%%%%%%
\section{VCD}
%%%%%%%%%%%%%%%%%%%%%%%%%%%%%%%%%%%%%%%%%%%%%%%%%%%%%%%%%%%%%%%%%%%%%%%%%%%%%%%%

%%%%%%%%%%%%%%%%%%%%%%%%%%%%%%%%%%%%%%%%%%%%%%%%%%%%%%%%%%%%%%%%%%%%%%%%%%%%%%%%
\section{Example of an internal module implementation} %TODO
%%%%%%%%%%%%%%%%%%%%%%%%%%%%%%%%%%%%%%%%%%%%%%%%%%%%%%%%%%%%%%%%%%%%%%%%%%%%%%%%

%%%%%%%%%%%%%%%%%%%%%%%%%%%%%%%%%%%%%%%%%%%%%%%%%%%%%%%%%%%%%%%%%%%%%%%%%%%%%%%%
\section{Embedding MCU Information in Binaries}
%%%%%%%%%%%%%%%%%%%%%%%%%%%%%%%%%%%%%%%%%%%%%%%%%%%%%%%%%%%%%%%%%%%%%%%%%%%%%%%%

%%%%%%%%%%%%%%%%%%%%%%%%%%%%%%%%%%%%%%%%%%%%%%%%%%%%%%%%%%%%%%%%%%%%%%%%%%%%%%%%
\section{Core Definitions}
%%%%%%%%%%%%%%%%%%%%%%%%%%%%%%%%%%%%%%%%%%%%%%%%%%%%%%%%%%%%%%%%%%%%%%%%%%%%%%%%

%%
%% = eof =====================================================================
%%
