%%
%% Design Approach
%%
%% This file should be edited by user
%%

%%%%%%%%%%%%%%%%%%%%%%%%%%%%%%%%%%%%%%%%%%%%%%%%%%%%%%%%%%%%%%%%%%%%%%%%%%%%%%%%
\chapter{simavr Internals} \label{chapter:simavr}
%%%%%%%%%%%%%%%%%%%%%%%%%%%%%%%%%%%%%%%%%%%%%%%%%%%%%%%%%%%%%%%%%%%%%%%%%%%%%%%%

It is necessary to understand simavr's internals before going on to discuss
qsimavr's design.

simavr is a small cross-platform \ac{AVR} simulator written with simplicity and
hackability in mind. It is supported on Linux and OS X, but should run on any
platform with avr-libc support.

In the following sections, we will take a tour through simavr internals\footnote{
Most, if not all of the code examined in this chapter is taken directly from simavr.}.
Without further delay, let's jump right in and walk through a short demo.


%%%%%%%%%%%%%%%%%%%%%%%%%%%%%%%%%%%%%%%%%%%%%%%%%%%%%%%%%%%%%%%%%%%%%%%%%%%%%%%%
\section{simavr Example Walkthrough}
%%%%%%%%%%%%%%%%%%%%%%%%%%%%%%%%%%%%%%%%%%%%%%%%%%%%%%%%%%%%%%%%%%%%%%%%%%%%%%%%

The following program is taken from the board\_i2ctest simavr example. Minor
modifications have been made to focus on the essential section. Error handling
is mostly omitted in favor of readability.

\begin{lstlisting}
#include <stdlib.h>
#include <stdio.h>
#include <libgen.h>
#include <pthread.h>

#include "sim_avr.h"
#include "avr_twi.h"
#include "sim_elf.h"
#include "sim_gdb.h"
#include "sim_vcd_file.h"
#include "i2c_eeprom.h"
\end{lstlisting}

The actual simulation of the external \ac{EEPROM} component is located in
i2c\_eeprom.h. We will take a look at the implementation later on.

\begin{lstlisting}
avr_t * avr = NULL;
avr_vcd_t vcd_file;

i2c_eeprom_t ee;
\end{lstlisting}

avr is the main data structure. It encapsulates the entire state of the
core simulation, including register, \ac{SRAM} and flash contents, the \ac{CPU} state, the
current cycle count, callbacks for various tasks, pending interrupts, and more.

\lstinline|vcd_file| represents the file target for the \emph{value change dump} module. It
is used to dump the level changes of desired pins (or \ac{IRQ}'s in general) into a
file which can be subsequently viewed using utilities such as \emph{gtkwave}.

\lstinline|ee| contains the internal state of the simulated external \ac{EEPROM}.

\begin{lstlisting}
int main(int argc, char *argv[])
{
    elf_firmware_t f;
    elf_read_firmware("atmega1280_i2ctest.axf", &f);
\end{lstlisting}

The firmware is loaded from the specified file. Note that exactly the same file
can be executed on the \ac{AVR} hardware without changes. \ac{MCU} and frequency
information have been embedded into the binary and are therefore available in
\lstinline|elf_firmware_t|.

\begin{lstlisting}
    avr = avr_make_mcu_by_name(f.mmcu);
    avr_init(avr);
    avr_load_firmware(avr, &f);
\end{lstlisting}

The \lstinline|avr_t| instance is then constructed from the core file of the
specified \ac{MCU} and initialized. \lstinline|avr_load_firmware| copies the
firmware into program memory.

\begin{lstlisting}
    i2c_eeprom_init(avr, &ee, 0xa0, 0xfe, NULL, 1024);
    i2c_eeprom_attach(avr, &ee, AVR_IOCTL_TWI_GETIRQ(0));
\end{lstlisting}

\lstinline|AVR_IOCTL_TWI_GETIRQ| is a macro to retrieve the internal \ac{IRQ} of the \ac{TWI}
simulation. \ac{IRQ}'s are the main method of communication between simavr and
external components and are also used liberally throughout simavr internals.
Similar macros exist for other important \ac{AVR} parts such as the \ac{ADC}, \ac{IO} ports,
timers, etc.

\begin{lstlisting}
    avr->gdb_port = 1234;
    avr->state = cpu_Stopped;
    avr_gdb_init(avr);
\end{lstlisting}

This section sets up simavr's \ac{GDB} infrastructure to listen on port 1234. The
\ac{CPU} is stopped to allow \ac{GDB} to attach before execution begins.

\begin{lstlisting}
    avr_vcd_init(avr, "gtkwave_output.vcd", &vcd_file, 100000 /* usec */);
    avr_vcd_add_signal(
        &vcd_file,
        avr_io_getirq(avr, AVR_IOCTL_TWI_GETIRQ(0), TWI_IRQ_STATUS),
        8 /* bits */,
        "TWSR");
\end{lstlisting}

Next, a value change dump output is configured to track changes to the
\lstinline|TWI_IRQ_STATUS| \ac{IRQ}. The file may then be viewed using the \emph{gtkwave}
application.

\begin{lstlisting}
    int state = cpu_Running;
    while ((state != cpu_Done) && (state != cpu_Crashed))
        state = avr_run(avr);

    return 0;
}
\end{lstlisting}

Finally, we have reached the simple main loop. Each iteration executes one
instruction, handles any pending interrupts and cycle timers, and sleeps if
possible. As soon as execution completes or crashes, simulation stops and we
exit the program.

We will now examine the relevant parts of the \lstinline|i2c_eeprom| implementation.
Details have been omitted and only communication with the \lstinline|avr_t| instance are
shown.

\begin{lstlisting}
static const char * _ee_irq_names[2] = {
		[TWI_IRQ_MISO] = "8>eeprom.out",
		[TWI_IRQ_MOSI] = "32<eeprom.in",
};

void
i2c_eeprom_init(
		struct avr_t * avr,
		i2c_eeprom_t * p,
		uint8_t addr,
		uint8_t mask,
		uint8_t * data,
		size_t size)
{

    /* [...] */

	p->irq = avr_alloc_irq(&avr->irq_pool, 0, 2, _ee_irq_names);
	avr_irq_register_notify(p->irq + TWI_IRQ_MOSI, i2c_eeprom_in_hook, p);

    /* [...] */
}
\end{lstlisting}

First, the \ac{EEPROM} allocates its own private \ac{IRQ}s. The \ac{EEPROM} implementation
does not know or care to which simavr \ac{IRQ}'s they will be attached. It then
attaches a callback function (\lstinline|i2c_eeprom_in_hook|) to the \ac{MOSI} \ac{IRQ}. This
function will be called whenever a value is written to the \ac{IRQ}. The pointer to
the \ac{EEPROM} state p is passed to each of these callback function calls.

\begin{lstlisting}
void
i2c_eeprom_attach(
		struct avr_t * avr,
		i2c_eeprom_t * p,
		uint32_t i2c_irq_base )
{
	avr_connect_irq(
		p->irq + TWI_IRQ_MISO,
		avr_io_getirq(avr, i2c_irq_base, TWI_IRQ_MISO));
	avr_connect_irq(
		avr_io_getirq(avr, i2c_irq_base, TWI_IRQ_MOSI),
		p->irq + TWI_IRQ_MOSI );
}
\end{lstlisting}

The private \ac{IRQ}s are then attached to simavr's internal \ac{IRQ}s. This is called
chaining - all messages raised are forwarded to all chained \ac{IRQ}s.

\begin{lstlisting}
static void
i2c_eeprom_in_hook(
		struct avr_irq_t * irq,
		uint32_t value,
		void * param)
{
	i2c_eeprom_t * p = (i2c_eeprom_t*)param;

    /* [...] */

    avr_raise_irq(p->irq + TWI_IRQ_MISO,
            avr_twi_irq_msg(TWI_COND_ACK, p->selected, 1));

    /* [...] */
}
\end{lstlisting}

Finally, we've reached the \ac{IRQ} callback function. It is responsible for
simulating communications between simavr (acting as the \ac{TWI} master) and the
\ac{EEPROM} (as the \ac{TWI} slave). The \ac{EEPROM} state which was previously passed to
\lstinline|avr_irq_register_notify| is contained in the \lstinline|param| variable and cast back to
an \lstinline|i2c_eeprom_t| pointer for further use.

Outgoing messages are sent by raising the internal \ac{IRQ}. This message is then
forwarded to all chained \ac{IRQ}s.


%%%%%%%%%%%%%%%%%%%%%%%%%%%%%%%%%%%%%%%%%%%%%%%%%%%%%%%%%%%%%%%%%%%%%%%%%%%%%%%%
\section{Main Loop} \label{Main Loop}
%%%%%%%%%%%%%%%%%%%%%%%%%%%%%%%%%%%%%%%%%%%%%%%%%%%%%%%%%%%%%%%%%%%%%%%%%%%%%%%%

We will now take a closer look at the main loop implementation. Each call to
\lstinline|avr_run| triggers the function stored in the run member of the \lstinline|avr_t| structure
(\lstinline|avr->run|\footnote{Whenever \lstinline|avr| is mentioned in a code
section, it is assumed to be the main \lstinline|avr_t| struct.}).
The two standard implementations are \lstinline|avr_callback_run_raw| and
\lstinline|avr_callback_run_gdb|, located in sim\_avr.c. The essence of both function is
identical; since \lstinline|avr_callback_run_gdb| contains additional logic for \ac{GDB}
handling (network protocol, stepping), we will examine it further and point out
any differences to the the raw version. Several comments and irrelevant code
sections have been removed.

\begin{lstlisting}
void avr_callback_run_gdb(avr_t * avr)
{
    avr_gdb_processor(avr, avr->state == cpu_Stopped);

    if (avr->state == cpu_Stopped)
        return ;

    int step = avr->state == cpu_Step;
    if (step)
        avr->state = cpu_Running;
\end{lstlisting}

This initial section is \ac{GDB} specific. \lstinline|avr_gdb_processor| is responsible for
handling \ac{GDB} network communication. It also checks if execution has reached a
breakpoint or the end of a step and stops the \ac{CPU} if it did.

If \ac{GDB} has transmitted a step command, we need to save the state during the
main section of the loop (the \ac{CPU} ``runs'' for one instruction) and restore to
the ``StepDone'' state at on completion.

\begin{lstlisting}
    avr_flashaddr_t new_pc = avr->pc;

    if (avr->state == cpu_Running) {
        new_pc = avr_run_one(avr);
    }
\end{lstlisting}

We have now reached the actual execution of the current instruction. If the \ac{CPU}
is currently running, \lstinline|avr_run_one| decodes the instruction located in flash memory
(\lstinline|avr->flash|) and triggers all necessary actions. This can include setting the \ac{CPU}
state (SLEEP), updating the status register \ac{SREG}, writing or reading from memory
locations, altering the program counter PC, etc \ldots

Finally, the cycle counter (\lstinline|avr->cycle|) is updated and the new
program counter is returned.

\begin{lstlisting}
    if (avr->sreg[S_I] && !avr->i_shadow)
        avr->interrupts.pending_wait++;
    avr->i_shadow = avr->sreg[S_I];
\end{lstlisting}

This section ensures that interrupts are not triggered immediately when
enabling the interrupt flag in the status register, but with an (additional)
delay of one instruction.

\begin{lstlisting}
    avr_cycle_count_t sleep = avr_cycle_timer_process(avr);
    avr->pc = new_pc;
\end{lstlisting}

Next, all due cycle timers are processed. Cycle timers are one of the
most important and heavily used mechanisms in simavr. A timer allows scheduling
execution of a callback function once a specific count of execution cycles have
passed, thus simulating events which occur after a specific amount of time has
passed. For example, the \lstinline|avr_timer| module uses cycle timers to schedule timer
interrupts.

The returned estimated sleep time is set to the next pending event cycle (or a
hardcoded limit of 1000 cycles if none exist).

\begin{lstlisting}
    if (avr->state == cpu_Sleeping) {
        if (!avr->sreg[S_I]) {
            avr->state = cpu_Done;
            return;
        }
        avr->sleep(avr, sleep);
        avr->cycle += 1 + sleep;
    }
\end{lstlisting}

If the \ac{CPU} is currently sleeping, the time spent is simulated using the callback
stored in \lstinline|avr->sleep|. In \ac{GDB} mode, the time is used to listen for
\ac{GDB} commands, while the raw version simply calls usleep.

It is worth noting that
we have improved the timing behavior by accumulating requested sleep cycles until
a minimum of 200 usec has been reached. usleep cannot handle lower sleep times
accurately, which caused an unrealistic execution slowdown.

A special case occurs when the \ac{CPU} is sleeping while interrupts are turned off.
In this scenario, there is way of ever waking up. Therefore, execution is halted
gracefully.

\begin{lstlisting}
    if (avr->state == cpu_Running || avr->state == cpu_Sleeping)
        avr_service_interrupts(avr);
\end{lstlisting}

Finally, any immediately pending interrupts are handled. The highest priority
interrupt (this depends solely on the interrupt vector address) is removed from
the pending queue, interrupts are disabled in the status register, and the
program counter is set to the interrupt vector.

If the \ac{CPU} is sleeping, interrupts can be raised by cycle timers.

\begin{lstlisting}
    if (step)
        avr->state = cpu_StepDone;
}
\end{lstlisting}

Wrapping up, if the current loop iteration was a \ac{GDB} step, the state is set
such that the next iteration will inform \ac{GDB} and halt the \ac{CPU}.


%%%%%%%%%%%%%%%%%%%%%%%%%%%%%%%%%%%%%%%%%%%%%%%%%%%%%%%%%%%%%%%%%%%%%%%%%%%%%%%%
\section{Initialization}
%%%%%%%%%%%%%%%%%%%%%%%%%%%%%%%%%%%%%%%%%%%%%%%%%%%%%%%%%%%%%%%%%%%%%%%%%%%%%%%%

%%%%%%%%%%%%%%%%%%%%%%%%%%%%%%%%%%%%%%%%%%%%%%%%%%%%%%%%%%%%%%%%%%%%%%%%%%%%%%%%
\subsection{\lstinline|avr_t| Initialization}
%%%%%%%%%%%%%%%%%%%%%%%%%%%%%%%%%%%%%%%%%%%%%%%%%%%%%%%%%%%%%%%%%%%%%%%%%%%%%%%%

The \lstinline|avr_t| struct requires some somple initialization before it is
ready to be used by the main loop as discussed in section \ref{Main Loop}.

\lstinline|avr_make_mcu_by_name| fills in all details specific to an \ac{MCU}. This
includes settings such as memory sizes, register locations, available components,
the default \ac{CPU} frequency, etc \ldots

The \ac{MCU} definitions are located in the simavr/cores subdirectory of the simavr
source tree and are compiled conditionally depending on the the local avr-libc
support. A complete list of locally supported cores is printed by running simavr
without any arguments.

On successful completion, it returns a pointer to the \lstinline|avr_t| struct.

If \ac{GDB} support is desired, \lstinline|avr->gdb_port| must be set, and
\lstinline|avr_gdb_init| must be called to create the required data structures,
set the \lstinline|avr->run| and \lstinline|avr->sleep| callbacks, and listen
on the specified port. It is also recommended to initially stop the cpu
(\lstinline|avr->state = cpu_Stopped|) to delay program execution until it
is started manually by \ac{GDB}.

Further settings can now be applied manually (typical candidates are logging and
tracing levels).

%%%%%%%%%%%%%%%%%%%%%%%%%%%%%%%%%%%%%%%%%%%%%%%%%%%%%%%%%%%%%%%%%%%%%%%%%%%%%%%%
\subsection{Firmware}
%%%%%%%%%%%%%%%%%%%%%%%%%%%%%%%%%%%%%%%%%%%%%%%%%%%%%%%%%%%%%%%%%%%%%%%%%%%%%%%%

We now have a fully initialized \lstinline|avr_t| struct and are ready to load
code. This is accomplished using \lstinline|avr_read_firmware|, which uses
elfutils to decode the \ac{ELF} file and read it into an \lstinline|elf_firmware_t|
struct and \lstinline|avr_load_firmware| to load its contents into the
\lstinline|avr_t| struct.

Besides loading the program code into \lstinline|avr->flash| (and \ac{EEPROM} contents
into \lstinline|avr->eeprom|, if available), there are several useful extended
features which can be embedded directly into the \ac{ELF} file.

The target \ac{MCU}, frequency and voltages can be specified in the \ac{ELF} file by using the
\lstinline|AVR_MCU| and \lstinline|AVR_MCU_VOLTAGES| macros provided by
avr\_mcu\_section.h:

\begin{lstlisting}
#include "avr_mcu_section.h"
AVR_MCU(16000000 /* Hz */, "atmega1280");
AVR_MCU_VOLTAGES(3300 /* milliVolt */, 3300 /* milliVolt */, 3300 /* milliVolt */);
\end{lstlisting}

\ac{VCD} traces can be set up automatically. The following code will create an 8-bit
trace on the UDR0 register, and a trace masked to display only the UDRE0 bit of
the UCSR0A register.

\begin{lstlisting}
const struct avr_mmcu_vcd_trace_t _mytrace[]  _MMCU_ = {
    { AVR_MCU_VCD_SYMBOL("UDR0"), .what = (void*)&UDR0, },
    { AVR_MCU_VCD_SYMBOL("UDRE0"), .mask = (1 << UDRE0), .what = (void*)&UCSR0A, },
};
\end{lstlisting}

Several predefined commands can be sent from the firmware to simavr during program execution.
At the time of writing, these include starting and stopping \ac{VCD} traces, and putting
UART0 into loopback mode. An otherwise unused register must be specified
to listen for command requests. During execution, writing a command to this
register will trigger the associated action within simavr.

\begin{lstlisting}
AVR_MCU_SIMAVR_COMMAND(&GPIOR0);

int main() {
    /* [...] */
    GPIOR0 = SIMAVR_CMD_\ac{VCD}_START_TRACE;
    /* [...] */
}
\end{lstlisting}

Likewise, a register can be specified for use as a debugging output. All bytes
written to this register will be output to the console.

\begin{lstlisting}
AVR_MCU_SIMAVR_CONSOLE(&GPIOR0);

int main() {
    /* [...] */
    const char *s = "Hello World\r";
    for (const char *t = s; *t; t++)
        GPIOR0 = *t;
    /* [...] */
}
\end{lstlisting}

Usually, UART0 is used for this purpose. The simplest debug output can be achieved
by binding \lstinline|stdout| to \lstinline|UART0| as described by the avr-libc
documentation \cite{libc}, and then using \lstinline|printf| and similar functions.
This alternate console output is provided in case using UART0 is not possible or desired.



%%%%%%%%%%%%%%%%%%%%%%%%%%%%%%%%%%%%%%%%%%%%%%%%%%%%%%%%%%%%%%%%%%%%%%%%%%%%%%%%
\section{Instruction Processing}
%%%%%%%%%%%%%%%%%%%%%%%%%%%%%%%%%%%%%%%%%%%%%%%%%%%%%%%%%%%%%%%%%%%%%%%%%%%%%%%%

We have now covered \lstinline|avr_t| initialization, the main loop, and loading
firmware files. But how are instructions actually decoded and executed? Let's
take a look at \lstinline|avr_run_one|, located in sim\_core.

The opcode is reconstructed by retrieving the two bytes located at
\lstinline|avr->flash[avr->pc]|. \lstinline|avr->pc| points to the \ac{LSB}, and
\lstinline|avr->pc + 1| to the \ac{MSB}. Thus, the full opcode is reconstructed with:

\begin{lstlisting}
uint32_t opcode = (avr->flash[avr->pc + 1] << 8) | avr->flash[avr->pc];
\end{lstlisting}

As we have seen, \lstinline|avr->pc| represents the byte address in flash memory.
Therefore, the next instruction is located at \lstinline|avr->pc + 2|. This
default new program counter may still be altered in the course of processing
in case of jumps, branches, calls and larger opcodes such as STS\cite{instructionset}.

Note also that the \ac{AVR} flash addresses are usually represented as word addresses
(\lstinline|avr->pc >> 1|).

Similar to the program counter, the spent cycles are set to a default value of 1.

The instruction and its operands are then extracted from the opcode and processed
in a large switch statement. The instructions themselves can be roughly categorized
into arithmetic and logic instructions, branch instructions, data transfer
instructions, bit and bit-test instructions and \ac{MCU} control instructions.

Processing these will involve a number of typical tasks:

\begin{itemize}
\item Status register modifications

The status register is stored in \lstinline|avr->sreg| as a byte array.
Most instructions alter the \ac{SREG} in some way, and convenience functions such as
\lstinline|get_compare_carry| are used to ease this task. Note that whenever the
firmware reads from \ac{SREG}, it must be reconstructed from \lstinline|avr->sreg|.

\item Reading or writing memory

\lstinline|_avr_set_ram| is used to write bytes to a specific address. Accessing
an \ac{SREG} will trigger a reconstruction similar to what has been discussed above.
\ac{IO} register accesses trigger any connected \ac{IO} callbacks and raise all associated
\ac{IRQ}s. If a \ac{GDB} watchpoint has been hit, the \ac{CPU} is stopped and a status report
is sent to \ac{GDB}. Data watchpoint support has been added by the author.

\item Modifying the program counter

Jumps, skips, calls, returns and similar instructions alter the program counter.
This is achieved by simply setting \lstinline|new_pc| to an appropriate value. Care must be
taken to skip 32 bit instructions correctly.

\item Altering \ac{MCU} state

Instructions such as SLEEP and BREAK directly alter the state of the simulation.

\item Stack operations

Pushing and popping the stack involve altering the stack pointer in addition
to the actual memory access. 
\end{itemize}

Upon conclusion, \lstinline|avr->cycle| is updated with the actual instruction
duration, and the new program counter is returned.

%%%%%%%%%%%%%%%%%%%%%%%%%%%%%%%%%%%%%%%%%%%%%%%%%%%%%%%%%%%%%%%%%%%%%%%%%%%%%%%%
\section{Interrupts}
%%%%%%%%%%%%%%%%%%%%%%%%%%%%%%%%%%%%%%%%%%%%%%%%%%%%%%%%%%%%%%%%%%%%%%%%%%%%%%%%

An interrupt is an asynchronous signal which causes the the \ac{CPU} to jump to
the associated \ac{ISR} and continue execution there. In the \ac{AVR} architecture,
the interrupt priority is ordered according to its place in the interrupt
vector table. When an interrupt is serviced, interrupts are disabled globally.

%%%%%%%%%%%%%%%%%%%%%%%%%%%%%%%%%%%%%%%%%%%%%%%%%%%%%%%%%%%%%%%%%%%%%%%%%%%%%%%%
\subsection{Data Structures}
%%%%%%%%%%%%%%%%%%%%%%%%%%%%%%%%%%%%%%%%%%%%%%%%%%%%%%%%%%%%%%%%%%%%%%%%%%%%%%%%

Let's take a look at how interrupts are represented in simavr:

\begin{lstlisting}
typedef struct avr_int_vector_t {
    uint8_t         vector;
    avr_regbit_t    enable;
    avr_regbit_t    raised;
    avr_irq_t       irq;
    uint8_t         pending : 1,
                    trace : 1,
                    raise_sticky : 1;
} avr_int_vector_t;
\end{lstlisting}

Each interrupt vector has an \lstinline|avr_int_vector_t|. \lstinline|vector| is
actual vector address, for example \lstinline|INT0_vect|. \lstinline|enable|
and \lstinline|raised| specify the \ac{IO} register index for, respectively, the
interrupt enable flag and the interrupt raised bit (again taking \lstinline|INT0|
as an example, enable would point to the \lstinline|INT0| bit in \lstinline|EIMSK|,
and raised to \lstinline|INTF0| in \lstinline|EIFR|. \lstinline|irq| is raised to
1 when the interrupt is triggered, and to 0 when it is serviced.

Usually, raised flags are cleared automatically upon interrupt servicing. However,
this does not count for all interrupts(notably, \lstinline|TWINT|).
\lstinline|raise_sticky| was introduced by the author to handle this special case.

Interrupt vector definitions are stored in an \lstinline|avr_int_table_t|,
\lstinline|avr->interrupts|.

\begin{lstlisting}
typedef struct  avr_int_table_t {
    avr_int_vector_t * vector[64];
    uint8_t         vector_count;
    uint8_t         pending_wait;
    avr_int_vector_t * pending[64];
    uint8_t         pending_w,
                    pending_r;
} avr_int_table_t, *avr_int_table_p;
\end{lstlisting}

\lstinline|pending_wait| stores the number of cycles to wait before servicing
pending interrupts. This simulates the real interrupt delay that occurs between
raising and servicing, and whenever interrupts are enabled
(and previously disabled).

\lstinline|pending| along with \lstinline|pending_w| and \lstinline|pending_r|
represents a ringbuffer of pending interrupts. Note that servicing an
interrupt removes the one with the highest priority.


%%%%%%%%%%%%%%%%%%%%%%%%%%%%%%%%%%%%%%%%%%%%%%%%%%%%%%%%%%%%%%%%%%%%%%%%%%%%%%%%
\subsection{Raising and Servicing Interrupts}
%%%%%%%%%%%%%%%%%%%%%%%%%%%%%%%%%%%%%%%%%%%%%%%%%%%%%%%%%%%%%%%%%%%%%%%%%%%%%%%%

When an interrupt \lstinline|vector| is raised, \lstinline|vector->pending| is
set, \lstinline|vector| is added to the \lstinline|pending| \ac{FIFO} of
\lstinline|avr->interrupts|, and a non-zero \lstinline|pending_wait| time is
ensured. If the \ac{CPU} is currently sleeping, it is woken up.

As we've already covered in section \ref{Main Loop}, servicing interrupts is
only attempted if the \ac{CPU} is either running or sleeping. Additionally,
interrupts must be enabled globally in \ac{SREG}, and \lstinline|pending_wait|
(which is decremented on each \lstinline|avr_service_interrupts| call) must have
reached zero. The next pending vector with highest priority is then removed from
the pending ringbuffer and serviced as follows:

\begin{lstlisting}
if (!avr_regbit_get(avr, vector->enable) || !vector->pending) {
    vector->pending = 0;
\end{lstlisting}

If the specific interrupt is masked or has been cleared, no action occurs.

\begin{lstlisting}
} else {
    _avr_push16(avr, avr->pc >> 1);
    avr->sreg[S_I] = 0;
    avr->pc = vector->vector * avr->vector_size;
    avr_clear_interrupt(avr, vector);
}
\end{lstlisting}

Otherwise, the current program counter is pushed onto the stack. This illustrates
the difference between byte addresses (as used in \lstinline|avr->pc|) and
word addresses (as expected by the \ac{AVR} processor).
Interrupts are then disabled by clearing the I bit of the status register, and
the program counter is set to the \ac{ISR} vector. Finally, if
\lstinline|raise_sticky| is 0, the interrupt flag is cleared.

%%%%%%%%%%%%%%%%%%%%%%%%%%%%%%%%%%%%%%%%%%%%%%%%%%%%%%%%%%%%%%%%%%%%%%%%%%%%%%%%
\section{Cycle Timers}
%%%%%%%%%%%%%%%%%%%%%%%%%%%%%%%%%%%%%%%%%%%%%%%%%%%%%%%%%%%%%%%%%%%%%%%%%%%%%%%%

%%%%%%%%%%%%%%%%%%%%%%%%%%%%%%%%%%%%%%%%%%%%%%%%%%%%%%%%%%%%%%%%%%%%%%%%%%%%%%%%
\section{\acf{GDB}}
%%%%%%%%%%%%%%%%%%%%%%%%%%%%%%%%%%%%%%%%%%%%%%%%%%%%%%%%%%%%%%%%%%%%%%%%%%%%%%%%

%%%%%%%%%%%%%%%%%%%%%%%%%%%%%%%%%%%%%%%%%%%%%%%%%%%%%%%%%%%%%%%%%%%%%%%%%%%%%%%%
\section{\acf{IRQ}}
%%%%%%%%%%%%%%%%%%%%%%%%%%%%%%%%%%%%%%%%%%%%%%%%%%%%%%%%%%%%%%%%%%%%%%%%%%%%%%%%

%%%%%%%%%%%%%%%%%%%%%%%%%%%%%%%%%%%%%%%%%%%%%%%%%%%%%%%%%%%%%%%%%%%%%%%%%%%%%%%%
\section{\acf{IO}}
%%%%%%%%%%%%%%%%%%%%%%%%%%%%%%%%%%%%%%%%%%%%%%%%%%%%%%%%%%%%%%%%%%%%%%%%%%%%%%%%

%%%%%%%%%%%%%%%%%%%%%%%%%%%%%%%%%%%%%%%%%%%%%%%%%%%%%%%%%%%%%%%%%%%%%%%%%%%%%%%%
\section{\acf{VCD}}
%%%%%%%%%%%%%%%%%%%%%%%%%%%%%%%%%%%%%%%%%%%%%%%%%%%%%%%%%%%%%%%%%%%%%%%%%%%%%%%%

%%%%%%%%%%%%%%%%%%%%%%%%%%%%%%%%%%%%%%%%%%%%%%%%%%%%%%%%%%%%%%%%%%%%%%%%%%%%%%%%
\section{Example of an internal module implementation} %TODO
%%%%%%%%%%%%%%%%%%%%%%%%%%%%%%%%%%%%%%%%%%%%%%%%%%%%%%%%%%%%%%%%%%%%%%%%%%%%%%%%

%%%%%%%%%%%%%%%%%%%%%%%%%%%%%%%%%%%%%%%%%%%%%%%%%%%%%%%%%%%%%%%%%%%%%%%%%%%%%%%%
\section{Embedding \acs{MCU} Information in Binaries}
%%%%%%%%%%%%%%%%%%%%%%%%%%%%%%%%%%%%%%%%%%%%%%%%%%%%%%%%%%%%%%%%%%%%%%%%%%%%%%%%

%%%%%%%%%%%%%%%%%%%%%%%%%%%%%%%%%%%%%%%%%%%%%%%%%%%%%%%%%%%%%%%%%%%%%%%%%%%%%%%%
\section{Core Definitions}
%%%%%%%%%%%%%%%%%%%%%%%%%%%%%%%%%%%%%%%%%%%%%%%%%%%%%%%%%%%%%%%%%%%%%%%%%%%%%%%%

%%
%% = eof =====================================================================
%%
