%%
%% Setup Guide
%%
%% This file should be edited by user
%%

\chapter{Setup Guide} \label{chapter:setup}

This section provides instructions on how to retrieve, compile and install
\simavr and \qsimavr on the GNU/Linux operating system.

\section{\simavr} \label{section:setup_simavr}

\subsection{Getting the source code}

The official home of \simavr is \url{https://github.com/buserror-uk/simavr}.
Stable releases are published as git repository tags (direct downloads are
available at \url{https://github.com/buserror-uk/simavr/tags}). To clone a local
copy of the repository, run

\begin{verbatim}
git clone git://github.com/buserror-uk/simavr.git
\end{verbatim}

\subsection{Software Dependencies}

\emph{elfutils} is the only hard dependency at run-time. The name of this
package may differ from distro to distro. For example, in Ubuntu the required
package is called \emph{libelf-dev}.

At compile-time, \simavr additionally requires \emph{avr-libc} to complete its
built-in AVR core definitions. It is assumed that further standard
utilities (\emph{git}, \emph{gcc} or \emph{clang}, \emph{make}, etc \ldots) are
already present.

\simavr has been tested with the following software versions:

\begin{itemize}
\item Arch Linux x86\_64 and i686
\item elfutils 0.154
\item avr-libc 1.8.0
\item gcc 4.7.1
\item make 3.82
\end{itemize}

Furthermore, the board\_usb example depends on libusb\_vhci and vhci\_hcd. For
further details, see \emph{examples/board\_usb/README}. Note however that these
are not required for a fully working \simavr build.

\subsection{Compilation and Installation}

\simavr's build system relies on standard makefiles. The simplest compilation
boils down to the usual

\begin{verbatim}
make
make install
\end{verbatim}

As usual, there are several variables to allow configuration of the build
procedure. The most important ones are described in the following section:

\begin{itemize}
\item AVR\_ROOT

The path to the system's \emph{avr-libc} installation.

While the default value
should be correct for many systems, it may need to be set manually if the
message 'WARNING \ldots did not compile, check your avr-gcc
toolchain' appears during the build. For example, if iomxx0\_1.h is located at
/usr/avr/include/avr/iomxx0\_1.h, AVR\_ROOT must be set to /usr/avr.

\item CFLAGS

The standard compiler flags variable.

It may be useful to modify CFLAGS for easier debugging (in which case
optimizations should be disabled and debugging information enabled: -O0 -g).
Additionally adding -DCONFIG\_SIMAVR\_TRACE=1 enables extra verbose output and
extended execution tracing.
\end{itemize}

These variables may be set either directly in Makefile.common, or alternatively
can be passed to the make invocation (make AVR\_ROOT=/usr/avr DESTDIR=/usr
install).

Installation is managed through the usual
\begin{verbatim}
make install
\end{verbatim}

The DESTDIR variable can be used in association with the PREFIX variable to create a \simavr package. DESTDIR=/dest/dir
PREFIX=/usr installs to /dest/dir but keeps the package configured to the standard prefix
(/usr).

For development, we built and installed \simavr with the following procedure:

\begin{verbatim}
make clean
make AVR_ROOT=/usr/avr CFLAGS="-O0 -Wall -Wextra -g -fPIC \
  -std=gnu99 -Wno-sign-compare -Wno-unused-parameter"
make DESTDIR="/usr" install
\end{verbatim}

\section{\qsimavr} \label{section:setup_qsimavr}

\subsection{Getting the source code}

The \qsimavr source code is available from
\url{git://github.com/schuay/qsimavr.git}, with the homepage located at
\url{https://www.github.com/schuay/qsimavr}. Clone it using

\begin{verbatim}
git clone git://github.com/schuay/qsimavr.git
\end{verbatim}

\subsection{Software Dependencies}

Since \qsimavr depends on \simavr, it shares all of its dependencies. The only
additional requirements are \emph{qt}, \emph{cmake} and \emph{avr-gcc}.

\qsimavr has been tested with

\begin{itemize}
\item Arch Linux x86\_64 and i686
\item qt 4.8.2
\item cmake 2.8.9
\item avr-gcc 4.7.1
\end{itemize}

\subsection{Compilation and Installation}

\qsimavr follows standard cmake procedure and supports out-of-tree builds. To
generate makefiles, run

\begin{verbatim}
cmake /path/to/qsimavr/source
\end{verbatim}

To switch to a debug build, add -DCMAKE\_BUILD\_TYPE="Debug" to the cmake
invocation. Installation paths can be customized using -DCMAKE\_INSTALL\_PREFIX.
See the cmake documentation for further information.

The rest of the compilation and installation process consists of running

\begin{verbatim}
make
make install
\end{verbatim}

As usual, DESTDIR may be used for packaging. Other standard make variables also
apply.

%%
%% = eof =====================================================================
%%
